\documentclass{nature3}
\usepackage{graphicx}
\usepackage{float}
\usepackage{verbatim}
\usepackage{hyperref}
\usepackage{amsmath}
\usepackage{amssymb}
\usepackage{aas_macros_nature}
\usepackage{lineno}

\linespread{1.0}
\linenumbers % turn line numbering on or off

\newcommand{\starname}{TIC 141146667}

\newcommand{\farcm}{\mbox{\ensuremath{.\mkern-4mu^\prime}}}%    % fractional arcminute symbol: 0.'0
\newcommand{\farcs}{\mbox{\ensuremath{.\!\!^{\prime\prime}}}}%  % fractional arcsecond symbol: 0.''0

\newcommand{\kms}{\ensuremath{\rm km\,s^{-1}}}
\newcommand{\ms}{\ensuremath{\rm m\,s^{-1}}}

\renewcommand*{\thefootnote}{\fnsymbol{footnote}}

%%%%%%%%%%%%%%%%
% INSTITUTIONS %
%%%%%%%%%%%%%%%%
\newcommand{\carnegie}{Observatories of the Carnegie Institution for Science, Pasadena, CA 91101, USA}
%%%%%%%%%%%%%%%%

%%%%%%%%%%
% VALUES %
%%%%%%%%%%
% NOTE: might need to be ingested before submission 
\newcommand{\stteff}{YYYY}
\newcommand{\stagemyr}{40}
\newcommand{\periodhr}{3.930}


%%%%%%%%%%%%%%%%%%%%%%%%%%%%%%%%%%%%%%%%%%
%%%%%%%%%%%%%%%%%%%%%%%%%%%%%%%%%%%%%%%%%%

\title{A Plasma Torus Around a Young Low-Mass Star}

\begin{document}

\author{Luke G. Bouma$^{1,2}$}

\maketitle

\scriptsize
\begin{affiliations}
\item \carnegie
\item Carnegie Fellow
\end{affiliations}
\normalsize

%%%%%%%%%%%%%%%%%%%%%%%%%%%%%%%%%%%%%%%%%%%%%%%%%%%%%%%%%%%%%%%%%%%%%%%%%%%%%%%
%%%%%%%%%%%%%%%%%%%%%%%%%%%%%%%%%%%%%%%%%%%%%%%%%%%%%%%%%%%%%%%%%%%%%%%%%%%%%%%

\begin{abstract}
\normalfont
% v1: removed a sentence for wordcount.  v0 is under abstract_title.txt
Approximately one percent of red dwarfs younger than 100 million years
show structured, periodic optical light curves suggestive of
transiting clumps of opaque circumstellar material that corotate
with the star \cite{Rebull2016,Stauffer2017,Rebull2018,Bouma2024}.
The composition, origin, and even the existence of this material are
uncertain. The main alternative hypothesis is that these stars are
explained by complex distributions of dark starspots or bright
faculae distributed across their surfaces \cite{Koen2021}.  Here, we
present time-series spectroscopy and photometry of a \stagemyr\
million year old complex periodic variable (CPV), TIC~141146667. The
spectra show coherent sinusoidal Balmer emission at up to four times
the star's equatorial velocity, demonstrating the presence of
extended clumps of circumstellar plasma --- a plasma torus.  Given
that long-lived condensations of cool ($10^4$ K) plasma can persist
in the hot ($10^6$ K) coronae of stars with a wide range of masses
\cite{CollierCameron1989,Townsend2005,Dunstone2006,Petit2013,Waugh2022,Daley-Yates2024},
these data support the idea that such condensations can become
optically thick around the lowest-mass stars, although the exact
source of opacity remains unclear.
\end{abstract}

\maketitle

%%%%%%%%%%%%%%%%%%%%%%%%%%%%%%%%%%%%%%%%%%%%%%%%%%%%%%%%%%%%%%%%%%%%%%%%%%%%%%%
%%%%%%%%%%%%%%%%%%%%%%%%%%%%%%%%%%%%%%%%%%%%%%%%%%%%%%%%%%%%%%%%%%%%%%%%%%%%%%%

% Main text – up to 3,000 words, excluding abstract, Methods,
% references and figure legends.

\section{Main}
\label{sec:main}

%\subsection{Introduction}
M dwarfs, stars with masses below about half that of the Sun, are the
only type of star to offer near-term prospects for detecting the
atmospheres of rocky exoplanets with water on their surfaces (CITE).
Investment with JWST has proceeded accordingly (CITE CITE).  It is
therefore important to consider how the evolution of an M dwarf might
impact the evolution of its planets.  Previous work has established
that most M dwarfs host close-in planets (CITE), and that these
planets are often subject to long circumstellar disk lifetimes (CITE),
to large doses of UV radiation (CITE), and to a high incidence of
flares and coronal mass ejections (CITE).  However, despite excellent
work in these areas, the properties of the circumstellar plasma
and magnetospheric environments to which young, close-in exoplanets
are subject remain largely unexplored. 

One glaring example of our current ignorance is the complex periodic
variables (CPVs).  Figure~\ref{fig:lc} highlights the main object of
interest in this article, but over one hundred analogous objects have
now been discovered by K2 and TESS (CITE, CITE, CITE, CITE).  These CPVs are
defined by their highly structured and periodic optical light curves, 
and most are M dwarfs with rotation periods shorter than two days.
Within current sensitivity limits, none have primordial disks (CITE).
However, $\approx$3\% of stars a few million years old show this
behavior (CITE), and the observed fraction decreases to $\approx$0.3\%
by $\approx$150\,Myr (CITE).

The two leading hypotheses to explain the CPVs are either that we are
in the shadow of transiting clumps of circumstellar material that
corotate with the star (CITE, CITE, CITE), or that these stars
represent an extreme in naturally-occurring distributions of starspots
or faculae for young M dwarfs \cite{Koen2021}.  Currently, the main
argument against a starspot-only explanation invokes the timescales
and amplitudes of the sharpest photometric features (CITE).  However,
there has yet to be any direct independent evidence for the presence of
circumstellar material in any of these objects.  Since transiting
circumstellar clumps would geometrically imply an occcurrence rate a
few to ten times the observed rate (CITE), this phenomenon has the
potential to affect 10-30\% of M dwarfs during their early lives.

The dearth of evidence for circumstellar material around CPVs is
surprising given that separate studies of young BAFGKM stars have, for
decades, reported that stellar coronae contain both hot ($10^6$ K) and
cool ($10^4$ K) plasma. In particular, time-series spectroscopy has
shown periodic high-velocity absorption and emission in Balmer lines
such as H$\alpha$, implying long-lived, corotating clumps of cool
plasma (CITE, CITE, CITE).  Such clumps are forced into corotation by
the magnetic field, and the geometry of where the plasma can
accumulate is thought to be dictated by the magnetic field's topology.
For instance, a tilted dipole field tends to yield an accumulation
surface of a warped torus \cite{Townsend2005}, whereas in the limit of
a single strong discrete field line, accumulation occurs along a fixed
point \cite{Waugh2022}.  However, none of these stars have shown any
photometric anomalies (CITE), leaving open the issue of whether these
two separate areas of study have any direct connection.  Nonetheless,
CPVs do respond to sudden magnetic field changes: there are many
documented cases of otherwise long-lived eclipse features disappearing
immediately following stellar flares (CITE, CITE).

In this study, we present the first observations of corotating clumps
of cool plasma around a CPV.  We identified TIC~141146667 in previous
work \cite{Bouma2024} by searching the TESS two-minute data for stars
showing periodic variability with at least three sharp dips.  We
selected it from the resulting fifty high-quality CPVs for follow-up
observations because it was the brightest source for which a full
cycle could be observed in a half-night.  We observed it for five
hours on UT 2024-02-17 using the High Resolution Echelle Spectrometer
(HIRES; \cite{vogt_hires_1994}) on the Keck I 10m telescope, roughly
contemporaneous with TESS, which observed the star from UT 2024-02-05
to UT 2024-02-26 with a duty cycle of XX\%.  In detail, TESS was
finishing a data downlink during the HIRES observations, and
photometric data collection resumed three cycle periods (12 hours)
after the spectra were acquired.  
Extended Data Figure~\ref{fig:fulllc} shows the detailed photometric
behavior of the star before and after the exact epoch of observation;
the star remained sufficiently stable to not affect any of the
interpretation that follows.

\begin{figure}[!t]
  \centering
  \includegraphics[width=0.7\textwidth]{figures/f1.png}
  \caption[]{{\bf Figure 1 (Movie):  TIC~141146667 is a complex periodic
  variable (CPV).} For the best experience, please view the online movie
  available
  \href{https://lgbouma.com/movies/movie_TIC1411_flux_phase.mp4}{here},
  which spans a baseline of 5{,}784 cycles irregularly sampled over three
  years.  The TESS light curve is phased to the \periodhr\ hour period in
  groups of a few cycles per frame.  This is the period both of
  stellar rotation, and (we hypothesize) of corotating clumps of
  circumstellar material.  Raw data acquired with two minute
  sampling are in gray; black is their average.  Similar to other members
  of this class, the sharp photometric features persist for tens to
  thousands of rotational cycles. }
  \label{fig:lc}
\end{figure}


\subsection{Results}

\begin{figure}[!tp]
  \centering
  \includegraphics[width=0.99\textwidth]{figures/f2.png}
  \caption[]{{\bf Figure 2 (Movie):}  
  Hydrogen emission from circumstellar plasma orbiting TIC 141146667.
  {(\bf TODO)}For the best experience, please view the online movie
  available
  \href{https://lgbouma.com/movies/TIC141146667_sixpanel.mp4}{here}.
  {\bf Panel a:} TESS light curve from UT 2024-02-05 to UT
  2024-02-26 folded on the \periodhr\ hour period.  Black points are
  averaged; gray are the raw data.
  {\bf Panel b:} Keck/HIRES H$\alpha$ spectra
  acquired on UT 2024-02-17.  The continuum is set to unity, and the
  darkest color is set at twice the continuum to accentuate emission
  outside the line core ($|v/v_{\rm eq}|>1$, for $v_{\rm eq}$=130\,\kms).
  While emission in the line core originates in the stellar
  chromosphere, the sinusoidal emission features are most readily
  described by a warped plasma torus.
  {\bf Panel c:} Individual epochs of Panel b, visible in the
  online movie.  The dotted line shows a time-averaged spectrum,
  $f_{\langle t \rangle}$.
  {\bf Panel d:} As in Panel a, but overplotting the
  median-normalized H$\alpha$ light curve at $|v/v_{\rm eq}|<1$.
  {\bf Panel e:} As in Panel b, after subtracting the time-averaged
  spectrum. In addition to circumstellar emission, the line core shows
  absorption during the plasma clump transits.  The asymmetric stretch
  is set to match the dynamic range of the data.
  {\bf Panel f:} Individual epochs of Panel e, visible in the online
  movie.}
  \label{fig:spec}
\end{figure}

Figure~\ref{fig:spec} shows the available photometry and spectroscopy from
February 2024.
While the star's photometric shape changed from February 2022,
its behavior remained complex:
the average photometric signal shows a smooth brightening over 45\% of
the period, followed by a complex eclipse-like feature spanning 55\%
of the period.
This photometric eclipse shows two to three local minima, 
and one to two local maxima at times near the spectroscopic
observation.

The novelty is the spectroscopy.  Emission from material well beyond
the star's equatorial velocity ($v_{\rm eq}$=130\,\kms) is visible by
eye in the raw spectrogram.
The emission is complex...
%TODO: what are the features you need to quantify / quote?

slightly less than half the period, 

While the two are not exactly simultaneous, 







\subsection{Discussion}

These data rule out ``starspot-only'' and ``dust-only''
origin scenarios for CPVs, instead supporting either a purely stellar
origin for the phenomenon or extrinsic scenarios involving long-lived
disks or outgassing rocky bodies capable of supplying sufficient gas.


%%%%%%%%%%%%%%%%%%%%%%%%%%%%%%%%%%%%%%%%%%%%%%%%%%%%%%%%%%%%%%%%%%%%%%%%%%%%%%%
%%%%%%%%%%%%%%%%%%%%%%%%%%%%%%%%%%%%%%%%%%%%%%%%%%%%%%%%%%%%%%%%%%%%%%%%%%%%%%%

\newpage
\begin{methods}

\renewcommand{\figurename}{Extended Data Figure}
\renewcommand{\tablename}{Extended Data Table}
\setcounter{table}{0}  
\setcounter{figure}{0}  


\subsection{Observations}

\begin{figure}[!b]
  \centering
  \includegraphics[width=0.99\textwidth]{figures/sf1.pdf}
  \caption{Detailed photometric evolution of TIC 141146667 near the
  epoch of spectroscopic observation (green). 
  {\bf Panel a}: Subset of TESS SAP\_FLUX acquired near time of
  Keck/HIRES observation.
  TESS downlinked data to the Deep Space Network from BTJD XXX to
  YYY, and was affected by scattered light from the Earth from BTJD
  3359.4 to 3360.15.
  %TODO: VERIFY!  Was it scattered light?  Or a legit flare?
  {\bf Panels b,c}: Folded light curve before and after spectroscopy.
  {\bf Panel d}: Zoom-in of Panel a, showing decreasing photometric
  scatter in the over three days (18 cycles).
  }
  \label{fig:fulllc}
\end{figure}

{\bf TESS:}

{\bf Keck/HIRES:}
We observed using the standard setup and reduction techniques of the
California Planet Survey \cite{Howard2010}.
Winds of 30 mph contributed to
1\farcs2$\pm$0\farcs2 seeing over the spectroscopic observations.  


\subsection{Data Reduction}

\subsection{Modeling the Emitting Clump}


%%%%%%%%%%%%%%%%%%%%%%%%%
% Supplementary Figures %
%%%%%%%%%%%%%%%%%%%%%%%%%

%%%%%%%%%%%%%%%%%%%%%%%%
% Supplementary Tables %
%%%%%%%%%%%%%%%%%%%%%%%%

% TEMPLATE: IRAS041
%
\begin{table}
    \centering
    \begin{tabular}{lcr}
    \hline 
    \hline
    Parameter & Host & Source \\
    \hline 
    \multicolumn{3}{c}{Identifiers} \\
    \hline
    TIC & 141146667 & TESS \\
    Gaia & todo & Gaia\ DR3 \\
    %2MASS & J04154278+2909597 & J04154269+2909558 & 2MASS \\
    %ALLWISE & J041542.77+290959.5 & ... & ALLWISE\\
    \hline
    \multicolumn{3}{c}{Astrometry} \\ 
    \hline
    $\alpha$ & todo & Gaia\ DR3 \\
    $\delta$ & todo & Gaia\ DR3 \\
    $\mu_{\alpha}$ (mas yr$^{-1}$ ) & todo & Gaia\ DR3 \\
    $\mu_{\delta}$ (mas yr$^{-1}$ ) & todo & Gaia\ DR3 \\
    $\pi$ (mas) & todo & Gaia\ DR3 \\
    \hline
    \multicolumn{3}{c}{Photometry} \\
    \hline
    $TESS$ (mag) & todo & TESS\ \\
    $G$ (mag) & todo & Gaia\ DR3 \\
    $G_{\rm BP}$ (mag) & todo & Gaia\ DR3\\
    $G_{\rm RP}$ (mag) & todo & Gaia\ DR3\\
    $J$ (mag) & todo & 2MASS\\
    $H$ (mag) & todo & 2MASS\\
    $K_s$ (mag) & todo & 2MASS\\
    $W1$ (mag) & todo & ALLWISE \\
    $W2$ (mag) & todo & ALLWISE \\
    $W3$ (mag) & todo & ALLWISE \\
    $W4$ (mag) & todo & ALLWISE \\
    \hline
    \multicolumn{3}{c}{Kinematics and Position} \\
    \hline
    $RV_{Bary}$ (km s$^{-1}$ ) & $13.35 \pm 3.39$ & Gaia\ DR3 \\
    $U$ (km s$^{-1}$ ) & & \\
    $V$ (km s$^{-1}$ ) & & \\
    $W$ (km s$^{-1}$ ) & & \\
    $X$ (pc)  & & \\
    $Y$ (pc)  & & \\
    $Z$ (pc) & & \\
    \hline
    \multicolumn{3}{c}{Physical Properties} \\
    \hline
    $P_{rot}$ (hours) & $3.930 \pm 0.XXX$ & This work \\ 
    $v \sin i_\star$(km s$^{-1}$) & todo & This work\\
    $i_\star$($^\circ$) & todo & This work \\
    $F_{bol}$ (erg cm$^{-2}$ s$^{-1}$ ) & todo & This work\\
    $T_{eff}$ (K) & todo & This work\\
    $A_V$ (mag) & todo & This work \\
    $R_\star$ ($R_{\odot}$) & todo & This work\\
    $L_\star$ ($L_{\odot}$)  & todo & This work\\
    $M_\star$ ($M_{\odot}$)  & todo & This work\\
    Age (Myr) & todo &  This work \\
    \hline
    \end{tabular}
    \caption{Properties of \starname.}
    \label{tab:stellarParameters}
\end{table}


\end{methods}

\bibliography{cpvbib.bib} % common bib file
\bibliographystyle{naturemagfixed}   


\begin{addendum}

\item[Acknowledgments] The author thanks X, Y, Z.
  L.G.B. was suported by...
	Acknowledge TESS...


%TC:ignore
%% Author Contribution
\item[Author Contributions] ...
%TC:endignore

\item[Data Availability] ...

\item[Competing Interests] The authors declare that they have no competing
financial interests.
 
\item[Correspondence] Correspondence and requests for materials should be
addressed to ...
 
\item[Code availability] We provide access to a GitHub repository including all
code created for the analysis of this project that is not already publicly
available.

\end{addendum}



\end{document}


