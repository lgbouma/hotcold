\documentclass{nature3}
\usepackage{graphicx}
\usepackage{float}
\usepackage{verbatim}
\usepackage{hyperref}
\usepackage{amsmath}
\usepackage{amssymb}
\usepackage{aas_macros_nature}
\usepackage{lineno}

\linespread{1.0}
\linenumbers % turn line numbering on or off

\newcommand{\starname}{TIC 141146667}

\newcommand{\farcm}{\mbox{\ensuremath{.\mkern-4mu^\prime}}}%    % fractional arcminute symbol: 0.'0
\newcommand{\farcs}{\mbox{\ensuremath{.\!\!^{\prime\prime}}}}%  % fractional arcsecond symbol: 0.''0

\newcommand{\kms}{\ensuremath{\rm km\,s^{-1}}}
\newcommand{\ms}{\ensuremath{\rm m\,s^{-1}}}

\renewcommand*{\thefootnote}{\fnsymbol{footnote}}

%%%%%%%%%%%%%%%%
% INSTITUTIONS %
%%%%%%%%%%%%%%%%
\newcommand{\carnegie}{Observatories of the Carnegie Institution for Science, Pasadena, CA 91101, USA}
%%%%%%%%%%%%%%%%

%%%%%%%%%%
% VALUES %
%%%%%%%%%%
% NOTE: might need to be ingested before submission 
\newcommand{\stteff}{YYYY}
\newcommand{\stagemyr}{40}
\newcommand{\periodhr}{3.930}


%%%%%%%%%%%%%%%%%%%%%%%%%%%%%%%%%%%%%%%%%%
%%%%%%%%%%%%%%%%%%%%%%%%%%%%%%%%%%%%%%%%%%

\title{A Plasma Torus Around a Young Low-Mass Star}

\begin{document}

\author{Luke G. Bouma$^{1,2}$}

\maketitle

\scriptsize
\begin{affiliations}
\item \carnegie
\item Carnegie Fellow
\end{affiliations}
\normalsize

%%%%%%%%%%%%%%%%%%%%%%%%%%%%%%%%%%%%%%%%%%%%%%%%%%%%%%%%%%%%%%%%%%%%%%%%%%%%%%%
%%%%%%%%%%%%%%%%%%%%%%%%%%%%%%%%%%%%%%%%%%%%%%%%%%%%%%%%%%%%%%%%%%%%%%%%%%%%%%%

\begin{abstract}
\normalfont
% v1: removed a sentence for wordcount.  v0 is under abstract_title.txt
Approximately one percent of red dwarfs younger than 100 million years
show structured, periodic optical light curves suggestive of
transiting clumps of opaque material corotating with the star
\cite{Rebull2016,Stauffer2017,Rebull2018,Bouma2024}.
The composition, origin, and even the existence of this material are
uncertain. The main alternative hypothesis is that these stars are
explained by complex distributions of dark starspots or bright
faculae distributed across their surfaces \cite{Koen2021}.  Here, we
present time-series spectroscopy and photometry of a \stagemyr\
million year old complex periodic variable (CPV), TIC~141146667. The
spectra show coherent sinusoidal Balmer emission at up to four times
the star's equatorial velocity, demonstrating the presence of
extended clumps of circumstellar plasma --- a plasma torus.  Given
that long-lived condensations of cool ($10^4$ K) plasma can persist
in the hot ($10^6$ K) coronae of stars with a wide range of masses
\cite{CollierCameron1989,Townsend2005,Dunstone2006,Petit2013,Waugh2022,Daley-Yates2024},
these data support the idea that such condensations can become
optically thick around the lowest-mass stars, although the exact
source of opacity remains unclear.
\end{abstract}

\maketitle

%%%%%%%%%%%%%%%%%%%%%%%%%%%%%%%%%%%%%%%%%%%%%%%%%%%%%%%%%%%%%%%%%%%%%%%%%%%%%%%
%%%%%%%%%%%%%%%%%%%%%%%%%%%%%%%%%%%%%%%%%%%%%%%%%%%%%%%%%%%%%%%%%%%%%%%%%%%%%%%

% Main text – up to 3,000 words, excluding abstract, Methods,
% references and figure legends.

\section{Main}
\label{sec:main}

%\subsection{Introduction}
M dwarfs, stars with masses below about half that of the Sun, are the
only type of star to offer near-term prospects for detecting the
atmospheres of rocky exoplanets with water on their surfaces
\cite{NAP26141}.  Investment with JWST has proceeded accordingly.  In
this context, it is therefore important to consider how the evolution
of an M dwarf might impact the evolution of its planets.  Previous
work has established that most M dwarfs host close-in planets
\cite{Dressing2015}, and that these planets are often subject to long
circumstellar disk lifetimes \cite{Ribas2015}, to large doses of UV
radiation \cite{France2013}, and to a high incidence of flares and
coronal mass ejections \cite{Gunther2020}.  However, despite excellent
work in these areas, the properties of the circumstellar plasma and
magnetospheric environments to which young, close-in exoplanets are
subject remain challenging to quantify. 

One glaring example of our current ignorance is the complex periodic
variables (CPVs).  Figure~\ref{fig:lc} highlights the main object of
interest in this article, but over one hundred analogous objects have
now been discovered by K2 and TESS 
\cite{Rebull2016,Stauffer2017,Rebull2018,Zhan2019,Rebull2020,Bouma2024}.
These CPVs are
defined by their highly structured and periodic optical light curves, 
and most are M dwarfs with rotation periods shorter than two days.
Within current sensitivity limits, none have primordial disks
\cite{Stauffer2017,Bouma2024}.
However, $\approx$3\% of stars a few million years old show this
behavior, and the observed fraction decreases to $\approx$0.3\%
by $\approx$150\,Myr \cite{Rebull2020}.

The two leading hypotheses to explain the CPVs are either that
transiting clumps of circumstellar material corotate with the star
\cite{Stauffer2017,Gunther2022,Bouma2024}, or that these stars
represent an extreme in naturally-occurring distributions of starspots
or faculae \cite{Koen2021}.  Currently, the main argument against a
starspot-only explanation invokes the timescales and amplitudes of the
sharpest photometric features.  However, no independent evidence has
yet been acquired for the presence of circumstellar material in these
objects.  Since transiting circumstellar clumps would geometrically
imply an intrinsic occurrence a few to ten times the observed rate,
the question of whether circumstellar clumps exist in these systems
has the potential to be applicable to 10-30\% of M dwarfs during their
early lives.

\begin{figure}[!t]
  \centering
  \includegraphics[width=0.7\textwidth]{figures/f1.png}
  \caption[]{{\bf Figure 1 (Movie):  TIC~141146667 is a complex periodic
  variable (CPV).} For the best experience, please view the online movie
  available
  \href{https://lgbouma.com/movies/movie_TIC1411_flux_phase.mp4}{here},
  which spans a baseline of 5{,}784 cycles irregularly sampled over three
  years.  The TESS light curve is phased to the \periodhr\ hour period in
  groups of a few cycles per frame.  This is the period both of
  stellar rotation, and (we hypothesize) of corotating clumps of
  circumstellar material.  Raw data acquired with two minute
  sampling are in gray; black is their average.  Similar to other members
  of this class, the sharp photometric features persist for tens to
  thousands of rotational cycles. }
  \label{fig:lc}
\end{figure}


The dearth of evidence for circumstellar material around CPVs is
surprising given that separate studies of young BAFGKM stars have, for
decades, reported that stellar coronae contain both hot ($10^6$ K) and
cool ($10^4$ K) plasma. In particular, time-series spectroscopy has
shown periodic high-velocity absorption and emission in Balmer lines
such as H$\alpha$, caused by long-lived, corotating clumps of cool
plasma \cite{CollierCameron1989,Donati2000,Dunstone2006,Skelly2008}.
Such clumps are forced into corotation by the magnetic field, and the
exact geometry of where the plasma can accumulate is dictated by the
magnetic field's topology.  For instance, a tilted dipole field tends
to yield an accumulation surface of a warped torus
\cite{Townsend2005}, whereas in the limit of a single strong field
line, accumulation occurs along the apexpoint furthest from the star
\cite{Waugh2022}.
However, none of these spectroscopic variables have shown any
photometric anomalies \cite{Bouma2024}, leaving open the issue of
whether these two areas of study have any direct connection.
Nonetheless, CPVs do respond to sudden magnetic field changes: the
otherwise long-lived eclipse features often disappear immediately
following stellar flares \cite{Stauffer2017,Bouma2024}.

In this study, we present the first observations of corotating clumps
of cool plasma around a CPV.  We identified TIC~141146667 in previous
work \cite{Bouma2024} by searching the TESS two-minute data for stars
showing periodic variability with at least three sharp dips per cycle.
We selected TIC~141146667 from the resulting fifty high-quality CPVs for
spectroscopic observations because it was the brightest source for
which a full cycle could be observed in a half-night.  We observed it
for five hours on UT 2024-02-17 using the High Resolution Echelle
Spectrometer (HIRES; \cite{vogt_hires_1994}) on the Keck I 10m
telescope.  TESS observed the star from UT 2024-02-05 to UT 2024-02-26
with a duty cycle of XX\%.  TESS was finishing a data downlink during
the spectral observations, and photometric data collection resumed
three rotation cycles (12 hours) after the spectra were acquired.
Extended Data Figure~\ref{fig:fulllc} shows the detailed photometric
behavior of the star before and after the exact epoch of observation;
the star remained sufficiently stable to not affect any of the
interpretation that follows.


\section{Results}

\begin{figure}[!tp]
  \centering
  \includegraphics[width=0.99\textwidth]{figures/f2.png}
  \caption[]{{\bf Figure 2 (Movie):}  
  Hydrogen emission from circumstellar plasma orbiting TIC 141146667.
  {(\bf TODO)}For the best experience, please view the online movie
  available
  \href{https://lgbouma.com/movies/TIC141146667_sixpanel.mp4}{here}.
  {\bf Panel a:} TESS light curve from UT 2024-02-05 to UT
  2024-02-26 folded on the \periodhr\ hour period.  Black points are
  averaged; gray are the raw data.
  {\bf Panel b:} Keck/HIRES H$\alpha$ spectra
  acquired on UT 2024-02-17.  The continuum is set to unity, and the
  darkest color is set at twice the continuum to accentuate emission
  outside the line core ($|v/v_{\rm eq}|>1$, for $v_{\rm eq}$=130\,\kms).
  While emission in the line core originates in the stellar
  chromosphere, the sinusoidal emission features are most readily
  described by a warped plasma torus.
  {\bf Panel c:} Individual epochs of Panel b, visible in the
  online movie.  The dotted line shows a time-averaged spectrum,
  $f_{\langle t \rangle}$.
  {\bf Panel d:} As in Panel a, but overplotting the
  median-normalized H$\alpha$ light curve at $|v/v_{\rm eq}|<1$.
  {\bf Panel e:} As in Panel b, after subtracting the time-averaged
  spectrum. In addition to circumstellar emission, the line core shows
  absorption during the plasma clump transits.  The asymmetric stretch
  is set to match the dynamic range of the data.
  {\bf Panel f:} Individual epochs of Panel e, visible in the online
  movie.}
  \label{fig:spec}
\end{figure}

% photometry
Figure~\ref{fig:spec} shows the data from February 2024.  As expected
based on other CPVs \cite{Bouma2024}, the photometric shape of
TIC~141146667 evolved over the years following the 2022 discovery
data, while nonetheless remaining complex.  In February 2024, the
average photometric signal showed a gradual brightening over 45\% of
the period, followed by a complex eclipse-like feature spanning 55\%
of the period.  This eclipse feature shows two to three
local photometric minima, and one to two local maxima.
Its W-shape is suggestive of eclipse geometries seen in forward models
of warped plasma tori \cite{Townsend2008}.

% ** keck/hires: emission beyond v/veq>1
The spectroscopy shows emission well outside the star's equatorial
velocity ($v_{\rm eq}$=130\,\kms).  There are at least two distinct
emission components, each spaced half a cycle apart in phase.  The
first has clearer sinusoidal behaviour and is double-peaked, with peak
semi-amplitudes of $K_1$=2.1\,$v_{\rm eq}$ and 2.7\,$v_{\rm eq}$.
There is also significant non-periodic variability in the spectra: for
instance, the flux excesses from these two peaks begin with amplitudes
equal to 100\% of the continuum flux early in the observation
sequence, and both fall to 30\% by its end.  The component 180$^\circ$
opposite in phase is similarly only detected from $\phi$=0.2-1.0.
From $\phi$=0.2-0.5, this latter component appears connected to the
star in velocity space.  While its peak semi-amplitude of
$K_1$=3.9\,$v_{\rm eq}$ is achieved at both $\phi$=0.25 and 0.75, its
amplitude decreases from a 60\% excess over the continuum at the
beginning of the observation sequence to a 10\% excess by its end.
The sinusoidal period for all of these emission signals is consistent
with the photometric \periodhr\ hour period.  

These sinusoidal emission features require circumstellar clumps of
partially-ionized hydrogen to be corotating with the star.  The
velocity semi-amplitude of the sinusoids directly gives the distance
of these clumps from the stellar surface: 2.1-2.7\,$R_\star$ for the
closer clump, and 3.9\,$R_\star$ for the other.  This material's
motion, rather than being Keplerian, can only be explained by plasma
being dragged along with the rotating stellar magnetic field.  These
clumps transit in front of the star when passing from negative to
positive velocity.

% * keck/hires: behavior at v/veq<1
The behavior within the stellar H$\alpha$ line core, at $|\Delta v /
v_{\rm eq}|<1$, is even more complex than outside it.  For stars of
this age and spectral type, one would expect emission in the line core
to be generated in the stellar chromosphere, near the star's surface,
and then modulated by any occulting material capable of absorbing or
emitting in H$\alpha$.  In Figure~\ref{fig:spec}e, the behavior from
$\phi$=0.4-1.2 is most easy to interpret: from $\phi$=0.4-0.9, a hot
region first gradually crosses the stellar line profile, followed from
$\phi$=0.7-1.2 by the transit of a cool region.  Phases $\phi$$<$0.4
seem to show a mix of similar events, though the time sampling is
sufficiently coarse that the interpretation is less clear.  A final
exercise to quantify the behavior in the line core is shown in
Figure~\ref{fig:spec}d, where $f_{\rm H\alpha\ core}$ denotes the
summed flux at $|\Delta v / v_{\rm eq}|<1$.  Changes in the line core
flux are usually correlated with the broadband variability, except at
$\phi$=0.5, during the transit of the higher-velocity clump and the
occultation of the lower-velocity clump.



\section{Discussion}
% argue: how does this clarify wider phenomenology?  what general
% conclusions can be drawn?

Magetically-active, rapidly rotating stars with a wide range of masses
have been known to exhibit both sinusoidal emission features
\cite{Donati2000,Townsend2005,Dunstone2006,Skelly2008} as well as
sharp transient absorption features in their line cores
\cite{CollierCameron1989,CollierCameron1992,Cang2020} similar to those
in Figure~\ref{fig:spec}.  No such stars were previously known to show
complex light curves \cite{Bouma2024}.  The usual interpretation for
their spectroscopic variability comes from a loose analogy to
quiescent solar prominences and filaments, which are cool
condensations of plasma in the solar corona that can last days to
weeks \cite{VialEngvold2015}.  This plasma is called a prominence when
viewed in emission against the dark backdrop of space, and a filament
when viewed in absorption against the solar disk.  In our Sun's
magnetosphere, these condensations fall back to the solar surface
because gravity is stronger than any magnetic or centrifugal force
capable of sustaining them.  However for stars with magnetospheric
radii $R_{\rm m}$ that exceed their corotation radii $R_{\rm c}$, the
effective potential experienced by a plasma parcel can have a local minimum
outside $R_{\rm c}$, enabling the material to be sustained for
much longer timescales \cite{Petit2013,Daley-Yates2024}.  Generally
speaking, such material need neither transit, nor be optically thick.

Our Keck/HIRES observations are the first high-resolution time-series
spectra of a CPV, and they show circumstellar plasma clumps corotating
with TIC~141146667.  Characteristic densities and masses of these
clumps are $n \sim 10^{10}$\,cm$^{-3}$ and $M \sim 10^{14}$\,kg (see
Supplementary Methods Section~\ref{subsec:model}), a similar density
to solar prominences, but ten to one hundred times more massive.  This
observation rules out a ``starspot-only'' origin scenario for CPVs,
\cite{Koen2021} since such scenarios have no means of explaining
spectroscopic emission beyond the stellar disk.  Similarly, scenarios
in which the circumstellar material is made only of dust are also
ruled out.  While dust may be present, to explain the H$\alpha$
emission the circumstellar clumps must include plasma with a
significant population of hydrogen atoms in the $n$=3 excited state.
This plasma is undoubtedly sculpted by the star's magnetic field.
However, it could plausibly originate from three sites:
the star, an old and undetected disk, or outgassing rocky bodies.
This latter possibility would render CPVs as extrasolar analogs of
the Jupiter-Io plasma torus (CITE).

The other potential analog for the CPVs are the $\sigma$~Ori~E
variables, a rare subset of B stars that host which can trap outflowing stellar winds into
warped plasma tori \cite{Townsend2005,Townsend2008}.  These tori tend
to have dense antipodal accumulations of plasma sculpted by
tilted-dipole magnetic fields, and the transits of
these clumps are thought to produce the observed broadband photometric variability
through a combination of bound-free scattering \cite{Townsend2005} and
Thomson scattering \cite{Berry2022}.  For $\sigma$~Ori~E and almost
all of its analogs, the result is light curves that appear ``simple'',
resembling those of eclipsing binaries.  The two known
exceptions, HD~37776 and HD~64740, show complex light curves
resembling CPVs \cite{Mikulasek2020,Bouma2024} and have
spectropolarimetric magnetic field maps indicating strong
contributions from higher-order magnetic moments
\cite{Kochukhov2011,Shultz2018}.  There are two implications: firstly,
the complexity of CPVs may be a direct consequence of magnetic fields
with highly multipolar contributions.  Secondly, CPVs could be a
source of astrophysical false positives in photometric searches for
eclipsing binaries and transiting exoplanets around young
pre-main-sequence M dwarfs \cite{Johns-Krull2016,Bouma2020}.

Pressing issues for future work on CPVs include determining the
composition and origin of the circumstellar material, understanding
the exact role of the stellar magnetic field, and exploring the
implied space weather experienced by the close-in rocky exoplanets
that, statistically \cite{Dressing2015}, are likely to be near our
observed plasma clumps.

The material's composition -- either purely plasma, or else a dusty
plasma -- can be clarified by time-series optical and infrared
spectrophotometry.  While observations of CPVs in the optical have
previously shown chromaticity consistent with dust \cite{Gunther2022,Koen2023},
a gray opacity source such as electron scattering in a plasma
transiting over starspots could also produce chromatic features
\cite{Rackham2018}.  The composition and size distribution of any dust
that is present could be most easily resolved by measuring the
extinction curve for one or more CPVs from $\approx$1-10\,$\mu$m.  A
dust composition similar to debris from rocky bodies seen around white
dwarfs \cite{Reach2009} would indicate a rocky-body origin.  A 
composition closer to the ISM would be indicative of condensed dust in
an M dwarf wind, similar to that formed in the environments of more
evolved stars \cite{Marigo2008}.

The role of the star's magnetic field could be better understood
through new observations, and new theory.  From the theoretical
perspective, there is an urgent need for rigid-field
(magneto)-hydrodynamic modeling to go beyond previous work
\cite{Townsend2005,Townsend2008} and to explore the effects of
non-dipolar field contributions.
Fully time-dynamic MHD \cite{Daley-Yates2024} will offer the
capability of understanding the connection between the plasma and the
dust.  %TODO: need this sentence to be better
Observationally, optical spectropolarimetry has the potential to
assess both the field strength and topology.  A more direct probe
however might be to connected the recent work \cite{Kaur2024} showing
that CPVs are variable radio emitters, exhibiting emission components
that can be both persistent, as well as short-lived and highly
polarized.  This opens the prospects for detecting for radio waves
emitted through the electron cyclotron maser instability, which can
provide a direct measurement of the field strength at the site of the
emitting region.

It is currently unclear what, if any, relationship CPVs have to the
close-in exoplanets that exist around most M dwarfs
\cite{Dressing2015}.  However, 0.3-3\% of young M dwarfs show the CPV
phenomenon \cite{Rebull2020}, and the phenomenon seems to be caused by
circumstellar clumps of material transiting the star.  The implied
geometric correction suggests that an appreciable minority (3-30\%) of
M dwarfs -- the rapidly rotating ones with centrifugal magnetospheres
-- have similar circumstellar environments to the CPVs.




%%%%%%%%%%%%%%%%%%%%%%%%%%%%%%%%%%%%%%%%%%%%%%%%%%%%%%%%%%%%%%%%%%%%%%%%%%%%%%%
%%%%%%%%%%%%%%%%%%%%%%%%%%%%%%%%%%%%%%%%%%%%%%%%%%%%%%%%%%%%%%%%%%%%%%%%%%%%%%%

\newpage
\begin{methods}

\renewcommand{\figurename}{Extended Data Figure}
\renewcommand{\tablename}{Extended Data Table}
\setcounter{table}{0}  
\setcounter{figure}{0}  


\section{Observations}

\begin{figure}[!b]
  \centering
  \includegraphics[width=0.99\textwidth]{figures/sf1.pdf}
  \caption{Detailed photometric evolution of TIC 141146667 near the
  epoch of spectroscopic observation (green). 
  {\bf Panel a}: Subset of TESS SAP\_FLUX acquired near time of
  Keck/HIRES observation.
  TESS downlinked data to the Deep Space Network from BTJD XXX to
  YYY, and was affected by scattered light from the Earth from BTJD
  3359.4 to 3360.15.
  %TODO: VERIFY!  Was it scattered light?  Or a legit flare?
  {\bf Panels b,c}: Folded light curve before and after spectroscopy.
  {\bf Panel d}: Zoom-in of Panel a, showing decreasing photometric
  scatter in the over three days (18 cycles).
  }
  \label{fig:fulllc}
\end{figure}

{\bf TESS:}

{\bf Keck/HIRES:}
We observed using the standard setup and reduction techniques of the
California Planet Survey \cite{Howard2010}.
Winds of 30 mph contributed to
1\farcs2$\pm$0\farcs2 seeing over the spectroscopic observations.  


\section{Data Reduction}

\section{Modeling the Emitting Clump}
\label{subsec:model}

% * mass estimate
The density and mass of the material...


%%%%%%%%%%%%%%%%%%%%%%%%%
% Supplementary Figures %
%%%%%%%%%%%%%%%%%%%%%%%%%

%%%%%%%%%%%%%%%%%%%%%%%%
% Supplementary Tables %
%%%%%%%%%%%%%%%%%%%%%%%%

% TEMPLATE: IRAS041
%
\begin{table}
    \centering
    \begin{tabular}{lcr}
    \hline 
    \hline
    Parameter & Host & Source \\
    \hline 
    \multicolumn{3}{c}{Identifiers} \\
    \hline
    TIC & 141146667 & TESS \\
    Gaia & todo & Gaia\ DR3 \\
    %2MASS & J04154278+2909597 & J04154269+2909558 & 2MASS \\
    %ALLWISE & J041542.77+290959.5 & ... & ALLWISE\\
    \hline
    \multicolumn{3}{c}{Astrometry} \\ 
    \hline
    $\alpha$ & todo & Gaia\ DR3 \\
    $\delta$ & todo & Gaia\ DR3 \\
    $\mu_{\alpha}$ (mas yr$^{-1}$ ) & todo & Gaia\ DR3 \\
    $\mu_{\delta}$ (mas yr$^{-1}$ ) & todo & Gaia\ DR3 \\
    $\pi$ (mas) & todo & Gaia\ DR3 \\
    \hline
    \multicolumn{3}{c}{Photometry} \\
    \hline
    $TESS$ (mag) & todo & TESS\ \\
    $G$ (mag) & todo & Gaia\ DR3 \\
    $G_{\rm BP}$ (mag) & todo & Gaia\ DR3\\
    $G_{\rm RP}$ (mag) & todo & Gaia\ DR3\\
    $J$ (mag) & todo & 2MASS\\
    $H$ (mag) & todo & 2MASS\\
    $K_s$ (mag) & todo & 2MASS\\
    $W1$ (mag) & todo & ALLWISE \\
    $W2$ (mag) & todo & ALLWISE \\
    $W3$ (mag) & todo & ALLWISE \\
    $W4$ (mag) & todo & ALLWISE \\
    \hline
    \multicolumn{3}{c}{Kinematics and Position} \\
    \hline
    $RV_{Bary}$ (km s$^{-1}$ ) & $13.35 \pm 3.39$ & Gaia\ DR3 \\
    $U$ (km s$^{-1}$ ) & & \\
    $V$ (km s$^{-1}$ ) & & \\
    $W$ (km s$^{-1}$ ) & & \\
    $X$ (pc)  & & \\
    $Y$ (pc)  & & \\
    $Z$ (pc) & & \\
    \hline
    \multicolumn{3}{c}{Physical Properties} \\
    \hline
    $P_{rot}$ (hours) & $3.930 \pm 0.XXX$ & This work \\ 
    $v \sin i_\star$(km s$^{-1}$) & todo & This work\\
    $i_\star$($^\circ$) & todo & This work \\
    $F_{bol}$ (erg cm$^{-2}$ s$^{-1}$ ) & todo & This work\\
    $T_{eff}$ (K) & todo & This work\\
    $A_V$ (mag) & todo & This work \\
    $R_\star$ ($R_{\odot}$) & todo & This work\\
    $L_\star$ ($L_{\odot}$)  & todo & This work\\
    $M_\star$ ($M_{\odot}$)  & todo & This work\\
    Age (Myr) & todo &  This work \\
    \hline
    \end{tabular}
    \caption{Properties of \starname.}
    \label{tab:stellarParameters}
\end{table}


\end{methods}

\bibliography{cpvbib.bib} % common bib file
\bibliographystyle{naturemagfixed}   


\begin{addendum}

\item[Acknowledgments] The author thanks X, Y, Z.
  L.G.B. was suported by...
	Acknowledge TESS...


%TC:ignore
%% Author Contribution
\item[Author Contributions] ...
%TC:endignore

\item[Data Availability] ...

\item[Competing Interests] The authors declare that they have no competing
financial interests.
 
\item[Correspondence] Correspondence and requests for materials should be
addressed to ...
 
\item[Code availability] We provide access to a GitHub repository including all
code created for the analysis of this project that is not already publicly
available.

\end{addendum}



\end{document}


