% upon AAS submission
\documentclass[11pt,twocolumn,tighten,linenumbers]{aastex7}
%\documentclass[11pt,twocolumn,tighten,linenumbers,trackchanges]{aastex7}
% drafting / arxiv
%\documentclass[11pt,twocolumn,modern]{aastex7}
\turnoffedit

\usepackage{apjfonts}
\usepackage{url}
\usepackage{hyperref}
\usepackage{natbib}
\usepackage{amsmath,amstext,amssymb}
\usepackage[caption=false]{subfig} % for subfloat
\usepackage{caption}
\DeclareCaptionLabelFormat{moviefmt}{#1~#2\ (Movie)}
\usepackage{xcolor, fontawesome}
\usepackage{color}
\usepackage{enumitem}
\usepackage{rotating} % sidewaystable

\linespread{1.0}
\linenumbers % turn line numbering on or off

\newcommand{\kms}{\ensuremath{\rm km\,s^{-1}}}
\newcommand{\ms}{\ensuremath{\rm m\,s^{-1}}}

\renewcommand*{\thefootnote}{\fnsymbol{footnote}}

%%%%%%%%%%%%%%%%
% INSTITUTIONS %
%%%%%%%%%%%%%%%%
\newcommand{\carnegie}{Observatories of the Carnegie Institution for Science, Pasadena, CA 91101, USA}
\newcommand{\caltech}{Department of Astronomy, California Institute of Technology, Pasadena, CA 91125, USA}
%%%%%%%%%%%%%%%%

%%%%%%%%%%
% VALUES %
%%%%%%%%%%
% NOTE: might need to be ingested before submission 
\newcommand{\stteff}{YYYY}
\newcommand{\stagemyr}{40}
\newcommand{\periodhr}{3.930}


%%%%%%%%%%%%%%%%%%%%%%%%%%%%%%%%%%%%%%%%%%
%%%%%%%%%%%%%%%%%%%%%%%%%%%%%%%%%%%%%%%%%%
%%%%%%%%%%%%%%%%%%%%%%%%%%%%%%%%%%%%%%%%%%

\begin{document}

\title{A Plasma Torus Around a Young Low-Mass Star}

\correspondingauthor{Luke G. Bouma}

\received{--}
\revised{--}
\accepted{--}
\shorttitle{CPV Plasma Torus} 

\shortauthors{Bouma et al.}

\author[orcid=0000-0002-0514-5538,sname='Bouma']{Luke~G.~Bouma}
\altaffiliation{Carnegie Fellow; 51 Pegasi b Fellow}
\affiliation{\carnegie}
\affiliation{\caltech}
\email{lbouma@carnegiescience.edu}

%%%%%%%%%%%%%%%%%%%%%%%%%%%%%%%%%%%%%%%%%%%%%%%%%%%%%%%%%%%%%%%%%%%%%%%%%%%%%%%
%%%%%%%%%%%%%%%%%%%%%%%%%%%%%%%%%%%%%%%%%%%%%%%%%%%%%%%%%%%%%%%%%%%%%%%%%%%%%%%

\begin{abstract}
  A small fraction of red dwarfs younger than $\sim$100 million years
  show structured, periodic optical light curves suggestive of
  transiting opaque material that corotates with the star.  However, the
  composition, origin, and even the existence of this material are
  uncertain. The main alternative hypothesis is that these complex
  periodic variables (CPVs) are explained by complex distributions of
  bright or dark regions on the stellar surfaces.  Here, we present
  time-series spectroscopy and photometry of a rapidly-rotating
  ($P$=3.9\,hr) CPV, TIC\,141146667.  The spectra show sinusoidal
  time-varying H$\alpha$ emission at twice to four times the star's
  equatorial velocity, providing direct evidence for cool
  ($\lesssim$10$^4$\,K) plasma clumps trapped in corotation around a
  CPV.  These data support the idea that young, rapidly-rotating M
  dwarfs can sustain warped plasma tori, similar to massive magnetic
  stars.  It remains unclear whether these structures originate
  intrinsically from the star or are fed by external sources. Likewise,
  the mechanism by which their optically thick clumps arise is still
  unknown.  Rough estimates suggest $\gtrsim$10\% of M dwarfs host
  similar structures during their early lives.
\end{abstract}

\keywords{Circumstellar matter (241), Stellar magnetic fields (1610),
Stellar rotation (1629) Periodic variable stars (1213), Weak-line T
Tauri stars (1795)}


%%%%%%%%%%%%%%%%%%%%%%%%%%%%%%%%%%%%%%%%%%%%%%%%%%%%%%%%%%%%%%%%%%%%%%%%%%%%%%%
%%%%%%%%%%%%%%%%%%%%%%%%%%%%%%%%%%%%%%%%%%%%%%%%%%%%%%%%%%%%%%%%%%%%%%%%%%%%%%%

\section{Introduction}
\label{sec:intro}

Stars with masses below about half that of the Sun, M dwarfs, are the
only type of star to offer near-term prospects for detecting the
atmospheres of rocky exoplanets with surface water.  Community
investment with JWST is proceeding accordingly
\citep[][]{Redfield2024,TRAPPIST1JWSTCommunityInitiative2024}.  The
question of how M dwarfs influence their planets---especially the
retention of their atmospheres---has correspondingly grown in
importance.  Previous work has established
that most M dwarfs host close-in planets \citep{Dressing2015} that on
average are subject to long circumstellar disk lifetimes
\citep{Ribas2015}, to high doses of UV radiation \citep{France2016},
and to a high incidence of flares and coronal mass ejections
\citep{Feinstein2020}.  However, despite extensive work in these
areas, the plasma and magnetospheric environments that bathe young,
close-in exoplanets remain challenging to quantify.  Understanding
these environments is crucial because they directly impact atmospheric
retention and habitability of close-in exoplanets.

One example of our current ignorance is the complex periodic
variables.  While Figure~\ref{fig:lc} highlights the main
object of interest in this article, over one hundred analogous systems
have now been found by K2 and TESS
\citep{Rebull2016,Stauffer2017,Rebull2018,Zhan2019,Rebull2020,Stauffer2021,Popinchalk2023,Bouma2024}.
These CPVs are phenomenologically identified based on their
structured, periodic optical light curves; most are M dwarfs with
rotation periods shorter than two days.  Within current sensitivity
limits, none host disks \citep{Stauffer2017,Bouma2024}.  However,
$\approx$3\% of stars a few million years old show this complex
behavior, an observed fraction which decreases to
$\approx$0.3\% by $\approx$110\,Myr \citep{Rebull2020}.  CPVs can and
have been confused for transiting exoplanets
\citep{vanEyken2012,Johns-Krull2016,Bouma2020}.

The two leading hypotheses for explaining CPVs are either that
transiting clumps of circumstellar material corotate with the star
\citep{Stauffer2017,Gunther2022,Bouma2024}, or that these stars
represent an extreme in naturally-occurring distributions of starspots
or faculae \citep{Koen2021}.  The main argument against a
starspot-only explanation invokes the timescales and amplitudes of the
sharpest photometric features.  However, no independent evidence has
yet been acquired for the presence of any circumstellar material.  If
such material exists, then the geometric correction from the transit
probability would imply an intrinsic occurrence rate at least a few
times larger than the observed rate, suggesting that these clumps
could exist around $\gtrsim$10\% of M dwarfs during their early lives.

The dearth of evidence for circumstellar material around CPVs is
surprising given that separate studies of young stars have, for
decades, reported that stellar coronae contain both hot
($\gtrsim$$10^6$ K) and cool ($\lesssim$$10^4$ K) plasma. In
particular, time-series spectroscopy of stars with a wide range of
masses has shown periodic high-velocity absorption and emission in
Balmer lines such as H$\alpha$, interpreted as long-lived, corotating
clumps of cool plasma
\citep{CollierCameron1989,CollierCameron1992,Barnes2000,Donati2000,Dunstone2006,Skelly2008,Leitzinger2016,Cang2021}.
Such clumps are thought to be forced into corotation by the star's magnetic
field, and the exact geometry of where the plasma can accumulate is
dictated by the field's topology.  For instance, a magnetic dipole
field tilted with respect to the stellar spin axis yields
accumulations in a warped torus geometry \citep{Townsend2005}, whereas
in the limit of a single strong field line, accumulation occurs at the
line's apex, furthest from the star \citep{Waugh2022}.  To date, none
of these spectroscopic variables have shown any photometric anomalies
\citep{Bouma2024}, leaving open the issue of whether they are related
to CPVs.

\begin{figure}[!t]
  \centering
  \includegraphics[width=0.47\textwidth]{figures/f1.pdf}
  \vspace{-0.3cm}
  \captionsetup{labelformat=moviefmt,labelsep=colon}
	\caption{\textbf{TIC\,141146667 is a complex periodic variable (CPV).}  The
online movie,
  \href{https://lgbouma.com/movies/TIC141146667_20250116.mp4}{{\bf
  available here}},
  covers a baseline of 5{,}784 cycles irregularly sampled over three
  years.  The TESS light curve is phased to the \periodhr\ hour period
  in groups of three cycles per frame.  This is the period both of
  stellar rotation, and (we hypothesize) of corotating clumps of
  circumstellar material.  Raw data acquired at two minute dence
  are in gray; black averages to 100 points per cycle.  
  The sharp photometric features persist
  for tens to thousands of rotational cycles. }
  \label{fig:lc}
\end{figure}

In this study, we present the first spectroscopic detection of
corotating clumps of cool plasma around a CPV, TIC\,141146667.
Section~\ref{sec:obs} describes our observations;
Section~\ref{sec:results} presents the results; Section~\ref{sec:disc}
discusses their interpretation and highlights future directions.


\section{Observations}
\label{sec:obs}

\begin{table}
\small
\setlength{\tabcolsep}{2pt}
\centering
\caption{Selected system parameters for TIC\,141146667.}
\label{tab:params}
\begin{tabular}{llcc}
\hline \hline
Parameter & Description & Value & Source\\
\hline 
%
$T_{\rm eff}$\dotfill                   & Effective Temperature (K) \hspace{9pt}\dotfill                 & 2972 $\pm$ 40    & 1 \\
%
$R_\star$\dotfill                       & Stellar radius ($R_\odot$)\dotfill                             & 0.42$\pm$0.02    & 1 \\
%
Age                                     & Stellar age range (Myr)\dotfill                                & 35-150           & 2 \\
%
$M_\star$\dotfill                       & Stellar mass ($M_\odot$)\dotfill                               & 0.22$\pm$0.02    & 3 \\
%
$\gamma$\dotfill                        & Systemic radial velocity (\kms)\dotfill                        & 0.61 $\pm$ 1.47  & 4 \\
%
SpT\dotfill                             & Spectral Type\dotfill                                          & M5.5Ve           & 4 \\
%
$P_{\rm rot}$\dotfill                   & Photometric rotation period (hr)\dotfill                       & $3.930\pm 0.001$ & 5 \\
%
$v_{\rm eq}$\dotfill		                & Equatorial velocity \dotfill                                   &  130$\pm$4       & 6 \\
                                        & \hspace{3pt} ($2\pi R_\star/P_{\rm rot}$) (\kms)	             &                      \\
%
$v_{\rm eq}\sin{i_\star}$\dotfill		    & Projected rotational\dotfill                                   &  138$\pm$8       & 4 \\
                                        & \hspace{3pt} velocity (\kms)	                                 &                      \\
%
$v_{\rm break}$\dotfill		              & Breakup velocity \dotfill                                      &  316$\pm$16      & 6 \\
                                        & \hspace{3pt} ($G M_\star / R_\star$)$^{1/2}$ (\kms)	           &                      \\
%
$i_\star$\dotfill                       & Stellar inclination\dotfill                                    & 	$>$63           & 4 \\
                                        & \hspace{3pt}  2$\sigma$ lower limit (deg)	                     &                      \\
%
$d$\dotfill                             & Distance (pc)\dotfill                                          & $57.54 \pm 0.09$ & 7 \\
%
$R_{\rm c}$\dotfill		                  & Keplerian corotation\dotfill                                   & $1.82 \pm 0.10$  & 6 \\
                                        & \hspace{3pt} radius ($R_\star$)	                               &                      \\
%
$a_0$\dotfill                           & Mean inner clump (0)\hspace{9pt}\dotfill           &  2.07$\pm$0.04   & 4 \\
                                        & \hspace{3pt} orbital radius ($R_\star$)	                       &                      \\
%
$a_1$\dotfill                           & Mean inner clump (1)\hspace{9pt}\dotfill           &  2.88$\pm$0.10   & 4 \\
                                        & \hspace{3pt} orbital radius ($R_\star$)	                       &                      \\
%
$a_2$\dotfill                           & Mean outer clump\hspace{9pt}\dotfill           &  3.88$\pm$0.25   & 4 \\
                                        & \hspace{3pt} orbital radius ($R_\star$)	                       &                      \\
%
$\langle$EW$_{\rm H\alpha}$$\rangle$    & Time-averaged H$\alpha$ line core                              &  7.2 $\pm$ 0.2   & 4 \\ 
                                        & \hspace{3pt} equivalent width (\AA)	                           &                      \\
\hline
\end{tabular}
\begin{flushleft}
\footnotesize{ \textsc{NOTE}---
Provenances are:
1: SED fit \citep{Bouma2024}.
2: Gaia DR3 photometry shows the star is on the pre-main sequence,
   while the spectrum lacks lithium (Appendix~\ref{sec:stparams}).
3: PARSEC v1.2S \citep{Chen2014}.
4: Keck/HIRES (Appendix~\ref{subsec:halpha}).
5: TESS light curve.
6: Derived quantity.
7: Gaia DR3 geometric \citep{GaiaDR3}.
}
\end{flushleft}
\vspace{-0.5cm}
\end{table}

\begin{figure*}[!t]
  \centering
  \includegraphics[width=0.925\textwidth]{figures/sf1.pdf}
  \vspace{-0.2cm}
  \caption{{\bf Evolution of TIC\,141146667 during the Keck/HIRES
  observation (green bar).}  {\bf a,} TESS simple aperture photometry.
  Data gaps caused by light from Earth (BTJD 3356-3358.5) and Moon
  (BTJD 3359.5-3360.5) are visible.  Raw two minute
  data are in gray; black time-averages to ten minute sampling.  {\bf
  b-c,} Folded TESS light curve before and after spectroscopy.  Black
  now phase-averages to 100 points per
  cycle.  During BTJD 3352-3356, a state switch occurred near BTJD
  3353, and the dip at $\phi$$\approx$0.8 disappeared.  While the
  large flux decrement was present both before and after the HIRES
  sequence, its photometric shape evolved during the data gap.  The
  orange bar denotes times of spectroscopic transits for the inner two
  H$\alpha$ clumps observed with HIRES (see Figure~\ref{fig:spec}).}
  \label{fig:fulllc}
\end{figure*}

\begin{figure*}[!t]
  \centering
  \includegraphics[width=0.99\textwidth]{figures/f2.pdf}
  \vspace{-0.25cm}
  \captionsetup{labelformat=moviefmt,labelsep=colon}
	\caption{\textbf{Emission from circumstellar plasma orbiting
TIC\,141146667.}
  The online movie,
  \href{https://lgbouma.com/movies/TIC141146667_sixpanel.mp4}{{\bf
  available here}},
  shows the spectral evolution over five hours.
  {\bf a,} Average TESS light curve from 5 February 2024 to 26
  February 2024 folded on the \periodhr\ hour period.  Black
  points are phase-averaged; gray are the raw data.
  {\bf b,} Keck/HIRES H$\alpha$ spectra acquired on 17
  February 2024.  The continuum is set to unity, and the darkest color
  is at twice the continuum to accentuate emission outside the line
  core ($|v/v_{\rm eq}|>1$, for $v_{\rm eq}$=130\,\kms).  While
  emission in the line core originates in the star's chromosphere,
  the sinusoidal emission features are most readily described by a
  warped plasma torus.
  {\bf c,} Individual epochs of Panel b, visible in the
  online movie.  The dotted line shows the time-averaged spectrum,
  $f_{\langle t \rangle}$.
  {\bf d,} As in Panel a, but overplotting the
  median-normalized H$\alpha$ light curve at $|v/v_{\rm eq}|<1$.
  {\bf e,} As in Panel b, after subtracting the time-averaged
  spectrum.  The line core shows H$\alpha$ excesses and decrements
  advancing from the blue to red wings.
  The asymmetric color stretch
  is set to mirror the dynamic range of the data.
  {\bf f,} Individual epochs of Panel e, visible in the online
  movie.}
  \label{fig:spec}
\end{figure*}




We identified TIC\,141146667 in previous work \citep{Bouma2024} by
searching TESS two-minute data from 2018-2022 for highly
structured, periodic light curves \citep{Ricker2015}.  We chose the
star for spectroscopy because its brightness and rapid rotation enabled
an efficient search for variability in its line profiles.
As an apparently single pre-main sequence M dwarf, its properties are
typical for the CPV population (see Table~\ref{tab:params} and
Appendix~\ref{sec:stparams}).

We observed TIC\,141146667 ($V$=16.2) for five hours on
17~February~2024 using the High Resolution Echelle Spectrometer
(HIRES; \citealt{vogt_hires_1994}) on the 10\,m Keck I telescope.  The
observations spanned the second half of the night, from 17 February
2024 10:47 to 16:13 (UT).  The star's airmass spanned $z$=1.2-2.2, and
we opted for a fixed 15 minute cadence, except for a final 10 minute
exposure due to increasing sky brightness at sunrise.  We observed
without the iodine cell and used the C2 decker
(0$\farcs$86$\times$14$\farcs$0) in the red configuration, yielding a
spectral resolution $R$$\approx$45{,}000 ($\delta
v$$\approx$6.7\,\kms; $\delta v / v_{\rm eq}$$\approx$0.05).  We
binned the CCD readout by a factor of three in the spatial dimension,
yielding $\approx$1,000 photons (S/N=33) per pixel in the continuum at
6500\,\AA, at minimum airmass.  Strong winds contributed to
1$''$-1\farcs5 seeing over the night, but conditions were otherwise
favorable.  We reduced the echelleogram to a one-dimensional spectrum
using the standard techniques of the California Planet Survey
\citep{Howard2010}.  
%Figure~\ref{fig:spec} shows the result in the
%vicinity of H$\alpha$ without any additional processing.

TESS observed TIC\,141146667 ($T$=13.3) for six non-contiguous months
spanning 2019-2024.  TESS acquired these observations at two-minute
cadence during Sectors 41, 48 (TESS DDT039, PI: Kunimoto), and 75
(TESS Program G06030, PI: Bouma).   The 30-minute data during Sectors
14, 15, and 21 probably smeared sharp features over the star's 3.93\,hour
period (see \citealt{Gunther2022}).  The movie in
Figure~\ref{fig:lc} shows the two-minute data: 
like other CPVs, TIC\,141146667 maintains a fixed period over
timescales of years while its detailed photometric morphology evolves.
The nearest known star, TIC\,141146666 ($T$=14.5), is 25$''$ from
TIC\,141146667 and is photometrically stable in the TESS images;
crowding is not a concern.


We observed with Keck during Sector~75 in an attempt to obtain
simultaneous observations.   Figure~\ref{fig:fulllc} shows the result;
Earth passed within 25$^\circ$ of the TESS camera's boresight from BTJD
3356.0-3358.5, which caused a data gap that ended twelve hours after our
Keck/HIRES observations (green bar).  From BTJD 3359.4-3362.0, the Moon
then passed within 25$^\circ$ of the camera's boresight.  Based on the
level of scattered light in the optimal TIC\,141146667 aperture
\citep{Jenkins2016}, we manually masked out data in the TESS light curve
from 3359.40-3360.13; the remainder of the data during the lunar
approach were usable.  Small gaps from BTJD 3353.55-3353.77 and
3360.12-3360.33 were caused by data downlinks at the spacecraft's
perigee and apogee.

Although the data gap is unfortunate, Figure~\ref{fig:fulllc} shows that
both before and after the HIRES data were acquired, a large flux
decrement spanned roughly half of each cycle.  From BTJD 3352-3356, this
dip had two sharp local minima;  the minimum at $\phi$$\approx$0.8
decreased in depth following the flare at BTJD 3353, yielding a dip more
closely resembling an asymmetric ``V'' than a ``W''.  Similar CPV state
changes have been previously noted \citep{Stauffer2017,Bouma2024}.  The
photometric shape therefore evolved during the twelve cycles spanning
the 3356-3358 gap, since the average shape from 3358-3363 more closely
resembles the initial ``W''.  Nonetheless, the general photometric
morphology---a small brightening over 45\% of the period, followed by a
complex flux dip spanning 55\% of the period---is similar before and
after the data gap.


\section{Results}
\label{sec:results}



Figure~\ref{fig:spec} compiles the TESS and HIRES data from February
2024.  The HIRES spectra show emission in H$\alpha$ beyond the star's
equatorial velocity, $v_{\rm eq}$, of 130\,\kms.  There are at least
two distinct emission components, separated by 180$^\circ$ in phase.
The inner component at lower velocities has clearer sinusoidal
behaviour in time and is double-peaked, with peak semi-amplitudes of
2.07\,$v_{\rm eq}$ and 2.88\,$v_{\rm eq}$, measured following the
procedure in Appendix~\ref{subsec:halpha}.  There is significant
non-periodic variability in the emissivity of this double-peaked
component: the flux excess begins with an amplitude 70\% that of the
continuum, and diminishes to 10\% by the end.  The higher-velocity
component 180$^\circ$ opposite in phase is detected only from
$\phi$=0.2-1.0.  From $\phi$=0.2-0.5, this outer component appears
connected to the star in velocity space.  While its peak
semi-amplitude of 3.88\,$v_{\rm eq}$ is achieved at both $\phi$=0.25
and 0.75, its amplitude similarly decreases from a 60\% excess over
the continuum at the beginning of the observation sequence to a 10\%
excess by its end.  
From the same procedure in Appendix~\ref{subsec:halpha},
we measured
the period for all three spectroscopic
emission components to be consistent with the photometric \periodhr\ hour
period, to within two minutes for the lower-velocity component.  

These sinusoidal emission features require circumstellar clumps of
partially-ionized hydrogen to corotate with the star.  Based on the
observed sinusoidal periods and velocities, this material's motion is
not controlled by gravitational attraction to the star; it is 
a plasma being dragged along with a rigidly rotating stellar magnetic
field.  The velocity semi-amplitude of the sinusoids gives the mean
distance of each clump from the stellar surface: 2.07\,$R_\star$ and
2.88\,$R_\star$ for the inner clumps, and 3.88\,$R_\star$ for the outer
clump.   These clumps transit in front of the star when passing from
negative to positive velocity.  The transits of the two inner clumps
last $\approx$22\% of each cycle, from $\phi$=-0.1 to $\phi$=+0.12 (see
Appendix~\ref{subsec:halpha}).  This spectroscopic transit coincides
with the latter half of the complex eclipse feature in the TESS data.

The H$\alpha$ line core is more complex.  At $|\Delta v / v_{\rm
eq}|<1$, most observed H$\alpha$ photons come from the star's
chromosphere; circumstellar material might also modulate the line
profile.  Figure~\ref{fig:spec}e suggests line core variability caused
by both bright and dark regions on the star's surface, superposed with
smaller-amplitude variability from the transiting circumstellar
material.  For instance, from $\phi$=0-0.3, the double-peaked emission
feature is visible when viewed both on and off-limb; this feature is
circumstellar in origin.  However, the large emission feature that
crosses the star from $\phi$=0.4-0.9 emits at an amplitude greater
than that observed from the circumstellar components, and it crosses
the stellar velocity surface at a speed that suggests it instead
originates from a chromospherically bright region on the star's
surface.  Similarly, from $\phi$=0.6-1.15 a 20\% deep absorption
feature slowly crosses the H$\alpha$ line profile.  This feature
suggests either a chromospherically dark region (e.g.~a starspot
group) crossing the star's surface, or an azimuthally extended
absorptive component to the circumstellar material which is not
visible in emission.  While the origin of the other bright and dark
streaks passing across the line core are similarly ambiguous, a final
exercise to quantify the behavior of the line core is shown in
Figure~\ref{fig:spec}d, where $f_{\rm H\alpha\ core}$ denotes the
summed flux at $|\Delta v / v_{\rm eq}|<1$.  This panel shows that
changes in the line core flux correlate with the broadband variability
throughout most of the light curve, except near $\phi$$\approx$0.5,
corresponding to the transit of the 3.9\,$R_\star$ clump and the
occultation of the lower-velocity clump.


\section{Discussion}
\label{sec:disc}

\subsection{Physical Properties of the Emitting Region}

% What can be inferred with confidence?

Our Keck/HIRES observations are the first reported time-series spectra
of a CPV, and they demonstrate that corotating circumstellar plasma
clumps exist around at least one such star.  More specifically, the
spectra require plasma with a significant population of hydrogen in the
$n$=3 excited state, with minimal evidence for higher-order excitations
(see Appendix~\ref{subsec:specvar}).  Most of this H$\alpha$ emission
originates in ``clumps'' with size comparable to the star; radial
``spokes'' or azimuthally-extended ``arcs'' for the emitting material
are ruled out by the $\approx$30\,\kms\ H$\alpha$ velocity dispersion
(see Appendix~\ref{subsec:halpha}).  The H$\alpha$ line luminosity
suggests characteristic number densities and masses for the gaseous
component of these clumps of $n_{\rm H} \sim 10^{11}$\,cm$^{-3}$ and
$M_{\rm gas} \sim 10^{17}$\,g (see Appendix~\ref{subsec:gas}).  Dust is
independently constrained to have a total mass $M_{\rm dust} <
10^{17}\,{\rm g}$ based on the lack of a WISE infrared excess; if one
assumes that the opacity in the TESS flux dips comes from dust, a lower
limit $M_{\rm dust} > 10^{15}\,{\rm g}$ follows (see
Appendix~\ref{subsec:dust}).  


\subsection{Clumps Within Warped Plasma Torii}

While the H$\alpha$ emission seems to mostly originate from clumps, the
complex flux dip in TIC\,141146667 spans roughly half of each cycle; in
CPVs more broadly, individual dips can last 5-50\% of each cycle, and
they are often distributed in longitude \citep{Bouma2024}.  This raises
the possibility that azimuthally-distributed material may be the norm
for CPVs, and that a warped torus may be a more accurate picture than a
clump.

\citet{Townsend2005} outlined the physics of how rapidly-rotating stars
with strong magnetic fields can support plasma tori.  When the
magnetospheric radii $R_{\rm m}$ of such stars exceeds their Keplerian
corotation radii $R_{\rm c}$, the effective potential experienced by
charged particles has a local minimum outside $R_{\rm c}$, which enables
charged material to accumulate in a torus
\citep{Petit2013,Daley-Yates2024}.  Warps can occur when there is
misalignment between the spin and magnetic axes.  In general, the
material in these centrifugally-supported magnetospheres need neither
transit nor be opaque in broadband optical light.  

In a follow-up study, \citet{Townsend2008} synthesized light curves for
these systems, assuming that the optical depth scaled linearly with
density.  They found that W-shaped eclipses, similar to those seen for
TIC\,141146667, can occur when the spin and magnetic axes are moderately
(15-45$^\circ$) misaligned.  The movies associated with their work are
available
online\footnote{\url{http://user.astro.wisc.edu/\~townsend/static.php?ref=rrm-movies\#Download\_Bundles}
last accessed 23 April 2025.}; for a $\omega/\omega_{\rm
c}$$\approx$0.5, edge-on system like TIC\,141146667, the photometric
eclipses have the correct shape, and the H$\alpha$ emission similarly
exhibits double-peaked behavior.  The plasma in this model accumulates
at antipodes 180$^\circ$ apart because the deepest local minima in the
effective potential exist along the line of intersection between the
rotational and magnetic planes (see Equation~22 of
\citealt{Townsend2005}).  

To summarize, the lines of evidence in favor of warped plasma tori for
TIC\,141146667 and CPVs more generally are that {\it i)}
Figure~\ref{fig:spec} shows two emitting plasma clumps separated
180$^\circ$ in phase, with double-peaked H$\alpha$ morphology at
quadrature; {\it ii)} the ``W'' shaped photometric eclipses are
predictions of such a model; {\it iii)} CPV dips last 5-50\% of each
cycle, suggesting an extended azimuthal distribution of circumstellar
material.

\subsection{Origin of CPV Photometric Variability and Astrophysical Analogs}

Microphysically, it is not obvious whether hydrogen alone can produce
the broadband chromatic flux variations seen in CPVs.  The
alternative idea is that charged dust could provide most of the
opacity \citep{Sanderson2023}.  For
TIC\,141146667, Figures~\ref{fig:fulllc}
and~\ref{fig:spec} show that the transits of the H$\alpha$ emitting
material only partially overlap the complex photometric modulation.
This implies that some of the complex photometric dips must involve
additional opacity sources or spatially distinct structures.

Independent of this opacity question, our observations show that
corotating clumps of cool plasma exist around a CPV.  While starspots do
contribute smooth signals to CPV photometric variability, the existence
of these clumps  would not be predicted by a ``starspot-only'' scenario
\citep{Koen2021}.  Scenarios in which the circumstellar material is made
only of dust are similarly ruled out.  

The circumstellar material -- either pure plasma or dusty plasma --
could originate either from the star or from an external source.
Plausible external sources include an undetected old disk, comets, or a
close-in exoplanet.  This latter scenario would make CPVs extrasolar
analogs of the Jupiter-Io torus (e.g.~\citealt{Bagenal1981}),
although with a very different composition.

The other analog for the CPVs are the $\sigma$~Ori~E variables, a rare
subset of B stars with radiatively-driven winds which accumulate into
warped plasma tori \citep{Townsend2005,Townsend2008}.  These tori tend
to have dense antipodal accumulations of plasma sculpted by
tilted-dipole magnetic fields, and the transits of these clumps produce
broadband photometric variability through bound-free scattering
\citep{Townsend2005} and Thomson scattering \citep{Berry2022}.  For
$\sigma$~Ori~E and almost all of its analogs, the result is simple
light curves that resemble those of eclipsing binaries, and time-dynamic
H$\alpha$ spectra similar to those in Figure~\ref{fig:spec}.  The two
known exceptions, HD~37776 and HD~64740, show complex light curves
resembling CPVs \citep{Mikulasek2020,Bouma2024} and have
spectropolarimetric magnetic field maps indicating strong contributions
from higher-order magnetic moments \citep{Kochukhov2011,Shultz2018}.
This suggests that the photometric complexity of CPVs may be related to
magnetic fields with highly multipolar contributions.  


\subsection{Connection to Previous Work}

Spectra of magnetically-active, rapidly rotating stars with a wide range
of masses have been previously observed to exhibit sinusoidal
time-varying Balmer emission
\citep{Donati2000,Townsend2005,Dunstone2006,Skelly2008}, similar to 
Figure~\ref{fig:spec}.  No such stars were previously known to show
complex light curves \citep{Bouma2024}.   One interpretation for the
spectroscopic variability of these stars, and that of the analogous
transient absorption features
\citep{CollierCameron1989,CollierCameron1992,Cang2020}, comes from an
analogy to quiescent solar prominences, cool condensations of plasma in
the solar corona that can last days to weeks \citep{VialEngvold2015}.
In the case of the Sun, these condensations fall back to the photosphere
because gravity is stronger than any magnetic tension or centrifugal
force capable of sustaining them.  However it has been understood at
least since work by \citet{Donati2000} that these ``prominence systems''
can be longitudinally extended, forming a trapped ring of plasma.  While
the issue of why these previously known spectroscopic variables do
not show complex photometric variability remains open, plausible
explanations include that they are not in the required transiting
geometry, or that they lack the necessary source of opacity.  What we
have in the case of TIC\,141146667 is a system that finally exhibits
both sets of phenomena.


\subsection{Future Work: Composition, Origin, Modelling}

Pressing issues include determining the composition and origin of the
circumstellar material, understanding the exact role of the stellar
magnetic field, and exploring the implied space weather experienced by
the close-in rocky exoplanets that, statistically \citep{Dressing2015},
are likely to be present in most CPV systems.

The material's composition -- either pure plasma or dusty plasma -- can
be clarified by time-series optical and infrared spectrophotometry.
While observations of CPVs in the optical suggest a chromaticity
consistent with dust \citep{Tanimoto2020,Gunther2022,Koen2023}, a gray
opacity source such as electron scattering in a plasma transiting over a
spotted background star might also produce chromatic features
\citep{Rackham2018}.  The composition and size distribution of any dust
that is present could be determined by measuring the extinction curve
for a sample of CPVs from 1-20\,$\mu$m.  Composition and size
distributions similar to debris from rocky bodies seen around white
dwarfs \citep{Reach2009} would suggest an extrinsic origin channel.
Compositions and sizes similar to the interstellar medium would suggest
that dust can condense out from M dwarf winds, similar to processes that
occur around more evolved stars \citep{Marigo2008}.

The role of the star's magnetic field could be better understood through
new observations, and new theory.  From the modeling perspective,
there is a need for rigid-field magnetohydrodynamic models to
go beyond previous work \citep{Townsend2005,Townsend2008,Krticka2022}
and to explore what field topologies explain the observed light
curve shapes.  In particular, dynamo simulations of
fully-convective M dwarfs have suggested that global-scale mean fields
might be confined to a single hemisphere \citep{Brown2020}; such fields
would yield accumulation surfaces quite different from those that have
previously been explored.  Observationally, spectropolarimetry has the
potential to assess both the field strength and topology.  An
independent constraint on magnetic field topology may also come from
radio observations, which have shown \citep{Kaur2024} that CPVs emit
variable radio signals, including persistent and polarized components.
Detecting radio emission produced by an electron cyclotron maser
instability (e.g.~\citealt{Callingham2021}) in particular would provide
a measurement of the field strength at the site of the emitting region,
which could help clarify the magnetosphere's structure.

It is currently unclear what, if any, relationship CPVs have to the
close-in rocky exoplanets that exist around most M dwarfs
\citep{Dressing2015}.  However, a few percent of young M dwarfs show
the CPV phenomenon \citep{Rebull2020}, and our data show that some of
the complex photometric features occur when clumps of circumstellar
material transit the star.  The suggested geometric correction implies
that an appreciable minority ($\gtrsim$10\%) of young M dwarfs -- the
rapidly rotating ones with centrifugal magnetospheres -- host
circumstellar environments similar to the CPVs.  Future studies that
combine spectroscopic, polarimetric, and multi-wavelength
observations, along with magnetohydrodynamic modeling, will be key to
understanding the complex environments of these young
stars.

\begin{acknowledgements}
  We thank M.~Jardine, A.~Weinberger, B.~Tofflemire,
  J.~Spake, J.~Winn, and L.~Hillenbrand for insightful conversations
  that significantly informed this work, and to A.~Howard and
  H.~Isaacson for their assistance reducing the HIRES spectra.

  This work was supported by the Carnegie Fellowship and by the
  Heising-Simons 51~Pegasi~b Fellowship.
  %
  Funding for the TESS mission is provided by NASA’s Science Mission
  directorate.
  TESS is a product of millions of hours of work by thousands of people;
  we thank the TESS team for their efforts to make the mission a
  continued success.
  The HIRES data were obtained at the Keck Observatory, which exists
  through a similar scale of community effort.
  We recognize the importance that the summit of Maunakea has always had
  within the indigenous Hawaiian community, and we are deeply grateful 
  for the opportunity to conduct observations from this mountain.
\end{acknowledgements}

\facilities{
  Gaia \citep{GaiaDR3},
  TESS \citep{Ricker2015},
  Keck:I (HIRES) \citep{vogt_hires_1994},
	2MASS \citep{Skrutskie2006},
	SDSS \citep{2000AJ....120.1579Y}.
}

\software{
  astropy \citep{astropy:2013,astropy:2018,astropy:2022},
  matplotlib \citep{matplotlib},
  numpy \citep{numpy},
  scipy \citep{scipy}.
}




%%%%%%%%%%%%%%%%%%%%%%%%%%%%%%%%%%%%%%%%%%%%%%%%%%%%%%%%%%%%%%%%%%%%%%%%%%%%%%%
%%%%%%%%%%%%%%%%%%%%%%%%%%%%%%%%%%%%%%%%%%%%%%%%%%%%%%%%%%%%%%%%%%%%%%%%%%%%%%%

\appendix

\section{Stellar Parameters}
\label{sec:stparams}

{\it Radial Velocity}---We measured the radial velocities of
TIC\,141146667 from our HIRES spectra using a pipeline that we
developed for rapidly rotating stars.  Our method is based on
template-matching against synthetic spectra produced by the PHOENIX
stellar atmosphere code \citep{Husser2013}.  We used the PHOENIX models
with solar metallicity and alpha element abundances, and calibrated
our pipeline using the standard stars described by \citet{Chubak2012}.
We used velocity standards spanning spectral types from G2 to M4
(Barnard's Star), irrespective of rotation rate.  We used
\texttt{barycorrpy} \citep{Kanodia2018} to calculate the velocity
corrections due to Earth's motion around the solar system barycenter
and due to Earth's daily rotation about its axis.  Our analysis code
reproduces the systemic velocities of known velocity standards
\citep{Chubak2012} with an RMS of 0.66\,\kms.

For TIC\,141146667, we measured the radial velocities using regions
near the K~I (7700\,\AA) resonance line and three TiO bandheads
(5160\,\AA, 5450\,\AA, and 5600\,\AA).  We selected these regions
because they provided the best matches between the synthetic and
observed spectra.  We then averaged the resulting redshift
measurements over each order to produce the final measurement.  We
used the scatter of resulting velocity measurements between orders to
assign the RV uncertainty at each epoch.  The uncertainty-weighted
mean systemic velocity over all epochs on 17~February~2024 was
$\gamma$=0.6$\pm$1.5\,\kms.  The relative radial velocities about this
mean are given in Table~\ref{tab:rv}.

{\it Viewing Orientation}---We fitted the rotational broadening of the
K~I (7700\,\AA) resonance line using the kernel suggested by
\citet{Gray2008}; taking the mean and standard deviation of the
resulting value over all epochs yielded $v_{\rm eq} \sin
i$=138$\pm$8\,\kms, consistent with the visual line broadening $\Delta
\lambda$$\approx$3\,\AA.  The star's equatorial velocity $v_{\rm eq}$
based on its apparent size and rotation period is 130$\pm$4\,\kms.
While this suggests that the viewing orientation could be nearly
edge-on, the formal constraint is rather weak, with $i$$>$63$^\circ$
at 2$\sigma$ (2.5$^{\rm th}$ percentile of the inclination posterior).

% from calc_max_amplitude_sinusoid_rv_timeseries.py
% from K_to_msini.py, assuming Mstar = 0.18

{\it No Evidence For Binarity}---Any periodicity in the radial
velocity time-series is ruled out at the rotation period for
semi-amplitudes above 2.85\,\kms\ (at 3$\sigma$ confidence).  This
sets an upper limit on the mass of any putative companion at the four
hour period of $m \sin i $$<$2.4\,$M_{\rm Jup}$.  Regarding possible
companions at wider separations, the Gaia DR3 renormalized unit weight
error (RUWE), a proxy for the goodness of fit for a single-source
astrometric model to the Gaia astrometry, is 1.23, within the usual
range for apparently single sources.  There are no resolved companions
in the Gaia DR3 point source catalog.  Finally, we checked the TESS
light curve for evidence of secondary photometric periods by
subtracting the mean CPV signal over each sector and performing a
phase-dispersion minimization analysis
\citep{Stellingwerf1978,2021zndo...1011188B}.  There were no secondary
periods in the TESS data.  Previous work \citep{Bouma2024} has shown
that about 30\% of CPVs show evidence for excess noise above the Gaia
single-source astrometric model, and about 40\% of CPVs show evidence
for unresolved binary companions based on the presence of secondary
photometric periods.  This agrees with analyses showing that
multi-periodic low-mass stars are generally unresolved binaries
\citep{Tokovinin2018}.  Overall, the CPV binary fraction seems
consistent with that for field M dwarfs \citep{Winters2019}, pointing
to a weak or non-existent connection between the CPV phenomenon and
(wide) binarity.  For TIC\,141146667 specifically, although we find no
evidence for stellar multiplicity, our data are minimally constraining
for the scenario of a low-luminosity companion
($F_1$/$F_2$$\lesssim$0.1) with apparent separation below 0$\farcs$1.


\begin{figure}[!t]
  \centering
  \includegraphics[width=0.48\textwidth]{figures/sf4.pdf}
  \caption{
    {\bf Spectral energy distribution} of broadband photometric magnitudes
    (filled markers) plotted over the best-fit BT-Settl stellar
    atmosphere model \citep{Allard2012} and the associated photometric
    predictions (empty markers).  This plot was made from an
    adaptation of \texttt{astroARIADNE} \citep{Vines2022}.  The
    photometry extends from the Gaia DR3 blue passband to WISE W3;
    the W4 passband (22 $\mu$m) did not yield a confident detection.
    This fit yields the star's temperature and size.  The lack of
    excess infrared flux relative to the photospheric model sets an
    upper limit on emission from circumstellar dust.
    }
  \label{fig:sed}
\end{figure}

{\it Effective temperature, radius, mass, and spectral
classification}---We adopted the star's effective temperature and
radius measured using the spectral energy distribution (SED) fitting
procedure described by \citep{Bouma2024}.  To summarize, this approach
used \texttt{astroARIADNE} \citep{Vines2022} with the BT-Settl stellar
atmosphere models \citep{Allard2012}, assuming the \citep{Asplund2009}
solar abundances and the \citep{Barber2006} water line lists.  This
approach fitted for the stellar effective temperature, radius,
reddening, surface gravity, metallicity, and distance by comparing the
measured broadband magnitudes against pre-computed model grids.
Specifically, we performed the fit using the broadband magnitudes from
Gaia DR2, APASS, 2MASS, SDSS, and WISE $W1$ and $W2$.  The resulting
best-fit SED is shown in Figure~\ref{fig:sed}.
This method has the most constraining power for the star's effective
temperature (2972 $\pm$ 40\,K) and radius (0.42 $\pm$
0.02\,$R_\odot$); the surface gravity, metallicity, and reddening are
only weakly constrained.  We determined the star's spectral type to be
M5.5Ve by visually comparing the HIRES spectra against the photometric
standards tabulated by \citep{Bochanski2007}.   This result agrees with
the effective temperature found from the SED fitting
\citep{Pecaut2013}.  We measured the equivalent width of the H$\alpha$
line by fitting a range of models to the time-averaged line profile
shown in Figure~\ref{fig:spec}, selecting the model that minimized the
Bayesian information criterion, and numerically integrating this best
fit model.  We found a sum of two Gaussians to be the preferred model;
our quoted result, EW$_{{\rm H}\alpha}$=7.2$\pm$0.2\,\AA, comes from
numerically integrating within $|\Delta v/v_{\rm eq}|<1$.  Integrating
over the entire line profile, including the broad wings, would yield
EW$_{{\rm H}\alpha}$=10.2$\pm$0.3\,\AA. Either value would classify
the star as a weak-lined T Tauri \citep{Briceno2019}.

Given the effective temperature, stellar radius, and age range
(35-150\,Myr) derived below, we then estimated the stellar mass by
interpolating against the PARSEC v1.2S isochrones \citep{Chen2014}.
Specifically, we used the distance metric defined in Equation~8 of
\citet{Bouma2024} to select the model mass closest to a given observed
temperature, radius, and age.  This exercise yielded a mass of
$M_\star$=$0.20\pm0.01$\,$M_\odot$ assuming an age of 35\,Myr, or a
mass of $0.25\pm0.01$\,$M_\odot$ assuming an age of 150\,Myr.  These
masses imply Keplerian corotation radii $R_{\rm
cr}/R_\star$=1.75$\pm$0.07 and $R_{\rm cr}/R_\star$=1.89$\pm$0.07,
respectively; this size scale is relevant because it is theoretically
expected to set the inner boundary at which corotating material might
accumulate (e.g.~\citep{Townsend2005,Daley-Yates2024}).  Our final
quoted $M_\star$ and $R_{\rm cr}$ values adopt the average of these
extremes and a quadrature sum of their statistical uncertainties; we
caution however that a more precise age would be needed to resolve the
systematic uncertainties in these parameters.


{\it Age: No Obvious Association Membership}---In
\citet{Bouma2024} we previously found that over 90\% of CPVs within
100\,pc are associated with
known young moving groups based on their positions and kinematics.
TIC\,141146667 is in the minority.  We calculated the probability of
TIC\,141146667 being part of any nearby known group using
BANYAN\,$\Sigma$ v1.2 \citep{Gagne2018}.  That particular model
classifies it as a field star at $>$99.9\% confidence.  We also
searched the local vicinity of TIC\,141146667 for neighbors with
similar projected on-sky velocities using \texttt{comove}
\citep{Tofflemire2021}.  This yielded no strong candidates for
co-moving stars with projected tangential velocities $\Delta v_{\rm
T}$$<$5\,\kms\ that share its isochronal youth.

\begin{figure}[!t]
	\begin{center}
		\subfloat{
			\includegraphics[width=0.47\textwidth]{figures/sf3.pdf}
		}
		\subfloat{
			\includegraphics[width=0.47\textwidth]{figures/sf2.pdf}
		}
	\end{center}
	\vspace{0.0cm}
  \caption{{\bf TIC\,141146667 age diagnostics.}
  {\it Left:} Dereddened Gaia DR3 color vs.~absolute magnitude diagram of
  TIC\,141146667 and comparison samples. 
  TIC\,141146667 is on the pre-main sequence; stars with the same color
  on the main sequence are $\approx$1.5\,magnitudes
  fainter.  The star's location in this diagram suggests an
  age of 30-150\,Myr. 
  {\it Right:} An upper limit on photospheric lithium for TIC\,141146667
  (green triangle) yields a lower bound on the star's age of
  $\gtrsim$20\,Myr.  Comparison stars are from the Gaia-ESO survey
  \citep{Jeffries2023}; rich clusters in each age bin include NGC\,2264
  (4.5\,Myr), $\lambda$~Ori (8.7\,Myr), ${\rm \gamma}^2$\,Vel
  (16.4\,Myr), NGC\,2547 (35.0\,Myr), IC\,2602 and IC\,2391 (42.0\,Myr),
  and NGC\,2451A (50.0\,Myr). }
  \label{fig:age_diagnostics}
\end{figure}

%  \label{fig:camd}
%  \label{fig:liew_population}
%\begin{figure}[!t]
%  \centering
%  \includegraphics[width=0.48\textwidth]{figures/sf2.pdf}
%  \caption{}
%\end{figure}




{\it Age: Isochrones}---The color and absolute magnitude of
TIC\,141146667 suggest that it is a pre-main sequence M dwarf, similar
to all other known CPVs \citep{Stauffer2017,Stauffer2021,Bouma2024}.
The star's proximity ($d$=58\,pc) and high galactic latitude
($b$=$+$53$^\circ$) yield negligible interstellar reddening along the
line of sight \citep{Green2019}.  Figure~\ref{fig:age_diagnostics}
shows the location of TIC\,141146667 in the color--absolute magnitude
diagram (CAMD) relative to young stellar populations including Upper
Scorpius (USco), IC~2602, and the Pleiades.  To make this diagram, we
adopted the USco members in the $\delta$~Sco and $\sigma$~Sco
sub-associations from \citet{Ratzenbock2023}, and the IC~2602 and
Pleiades members from \citet{Hunt2024}.  We assumed an average $V$-band
extinction $A_{\rm V}$=$\{0.12, 0.11, 0.10\}\,{\rm mag}$ for USco
\citep{Pecaut2016}, IC~2602 \citep{Hunt2024}, and the Pleiades
\citep{Hunt2024} respectively, and ages of 8\,Myr
\citep{Ratzenbock2023}, 40\,Myr \citep{Randich2018}, and 112\,Myr
\citep{Dahm2015} for each respective cluster.  We dereddened the
photometry using the extinction coefficients $k_X\equiv A_X/A_0$
tabulated in \citep{GaiaCollaboration2018}, assuming that $A_0 = 3.1
E(B-V)$.

Figure~\ref{fig:age_diagnostics} shows that TIC\,141146667 falls
between the USco and Pleiades sequences, and approximately overlaps
IC~2602.  However, the precision of the implied age is set by the
intrinsic scatter of these calibration sequences; the most luminous
stars in the Pleiades of the same color have a similar absolute
magnitude as TIC\,141146667.  Previous work by \citet{Stauffer2021} has
also noted that in the Gaia passbands, CPVs tend to be photometrically
redder and more luminous than single stars in any given cluster,
similar to other rapid rotators.  While this effect complicates any
attempt at age inference based on the Gaia photometry, it suggests
that the Pleiades may be a better comparison population than IC~2602.
We take the star's location in the color--absolute magnitude diagram
to suggest age bounds $t_{\rm CAMD}$$\sim$30-150\,Myr.


{\it Age: Lithium}---The depletion of lithium due to $^7$Li(p,
$\alpha$)$^4$He burning in the cores of low-mass stars has been
studied for over sixty years
\citep{Hayashi1963,Bildsten1997,Burke2004}.  \citet{Wood2023} provided a
recent overview: an abbreviated summary is that sufficiently cool and
young M dwarfs show the 6707.8\,\AA\ doublet in absorption,
$\gtrsim$10\% below their continua.  However unlike for Sun-like
stars, the continuum for M dwarfs is challenging to define due to
their molecular absorption.  We therefore attempted a lithium
measurement by constructing a wavelength-binned and Doppler-corrected
TIC\,141146667 spectrum, assigning its uncertainties based on the
measured scatter across the five hour dataset.  We then compared this
average spectrum against the nearest matching M6 template from
\citet{Bochanski2007}.  The data show a small depression near the
expected lithium wavelength, potentially consistent with the $\Delta
\lambda$$\approx$3\,\AA\ line broadening.  This feature nominally
yields ${\rm EW}_{\rm Li}$=71$^{+18}_{-13}$\,m\AA, where the
statistical uncertainties are evaluated using a bootstrap resampling
technique from the statistical uncertainties in the HIRES spectrum.
However, the systematic uncertainties associated with the continuum
normalization are likely comparable to the amplitude of this feature;
we therefore treat the result of this measurement as a $2\sigma$ upper
limit: ${\rm EW}_{\rm Li}$$<$107\,m\AA.


Despite ambiguity in the details, what can be stated with
confidence is that lithium is not abundant in the spectrum of
TIC\,141146667.  Figure~\ref{fig:age_diagnostics}
compares our upper limit against the equivalent width measurements
reported by \citet{Jeffries2023} based on the Gaia-ESO spectroscopic
survey; Figure~\ref{fig:hirescuts} shows the associated
raw spectra.  If the star were $\lesssim$20\,Myr old, at its
temperature we would expect to see lithium in abundance
($>$400\,m\AA).  Since we do not, we can set an empirical bound on the
lithium-derived age of $t_{\rm Li,emp}$$\gtrsim$20\,Myr.  The
\citet{Feiden2016} lithium isochrones provide a point for theoretical
comparison, and suggest that since $M_{\rm K}$$\approx$6.67\,mag,
$t_{\rm Li,th}$$\gtrsim$35\,Myr is the theoretical age at which
complete depletion occurs in a star with this luminosity (see
e.g.~Figure 7 from \citealt{Wood2023}).


{\it Age: Summary}---The main indicators for the youth of
TIC\,141146667 are {\it i)} that it is a complex periodic variable, and
{\it ii)} that it is 1.5 magnitudes brighter (four times more
luminous) than main sequence stars of the same color, while showing no
indicators for binarity.  Being a CPV suggests that the star is young
because a previous CPV search unbiased in age found 90\% of its
detections to be in $\lesssim$200\,Myr old clusters \citep{Bouma2024};
the remaining 10\% were not associated with any coeval population.
Similarly, studies of rotation in $\lesssim$100\,Myr clusters
serendipitously found $\approx$50-100 examples of the class
\citep{Rebull2016,Stauffer2017,Stauffer2018,Rebull2018,Zhan2019,Rebull2020,Stauffer2021,Rebull2022,Popinchalk2023},
whereas analogous studies of Praesepe and the Hyades did not report
any evidence for CPVs in a set of approximately one thousand
$\approx$700\,Myr stars \citep{Rebull2017,Douglas2019,Rampalli2021}.
Regarding the isochronal age constraint, the Pleiades (112\,Myr,
\citealt{Dahm2015}) shows a few stars of equal luminosity and the same
temperature, suggesting a photometric isochronal age upper limit
$\lesssim$150\,Myr.  The weak lithium absorption suggests an age of at
least 20\,Myr based on an empirical comparison using Gaia-ESO spectra,
or at least 35\,Myr based on the \citet{Feiden2016} isochrones.  These
considerations yield our adopted age range of 35-150\,Myr.




\section{Detailed Behavior of H$\alpha$: Model and Implications}
\label{subsec:halpha}

\begin{figure*}[!tp]
  \centering
  \includegraphics[width=0.5\textwidth]{figures/sf6.pdf}
  \caption{{\bf Time-variable fit to H$\alpha$ line profiles.}  {\it Left
  column:} Raw spectrum at each epoch $f_{\lambda}$ minus the
  time-averaged spectrum $f_{\langle t \rangle}$ (as in
  Figure~\ref{fig:spec}e).  Underplotted sinusoids are not fits; they
  are meant to guide the eye.  {\it Middle columns:} Model of emission
  from the inner clump (sum of two gaussians) and the outer clump
  (single gaussian), plotted over the data.  {\it Right column:}
  Residual of the left column after subtracting the sum of the two
  middle columns, leaving variability in the line core.
  Appendix~\ref{subsec:halpha} discusses the use of this model.  }
  \label{fig:halphamodel}
\end{figure*}

{\it A model for the time-dynamic H$\alpha$ spectrum}---While
Figure~\ref{fig:spec} shows that clumps of circumstellar material
exist around TIC\,141146667, there is value in quantifying the exact
orbital periods, velocities, and velocity dispersions of these clumps.
These quantities can constrain the physical dimensions of the emitting
region, and can also clarify whether the spectroscopic period agrees
with the photometric period.

Given that a full radiative transfer simulation was outside our scope,
we opted to construct a phenomenological model aimed at capturing the
emission from the circumstellar material.  We did this by fitting each
spectral epoch with a multi-component gaussian, after having
subtracted the time-average line profile as in Figure~\ref{fig:spec}e.
The results of this exercise are shown in
Figure~\ref{fig:halphamodel}; details in the implementation and
interpretation follow.

To implement this model, we assumed that the ``inner'' ($K_{\rm
inner}$$\approx$2.5\,$v_{\rm eq}$) clump would be well-fit by a sum of
two gaussians because it is visually double-peaked in the raw data
from $\phi$=0.15-0.35 and $\phi$=0.65-0.85 (Figure~\ref{fig:spec}b).
We assumed that the ``outer'' ($K_{\rm outer}$$\approx$3.9\,$v_{\rm
eq}$) clump would be better fit by a single gaussian, based on its
behavior from $\phi$=0.6-0.9.  Each gaussian component has three free
parameters at each spectral epoch: the mean $\mu$, standard deviation
$\sigma$, and amplitude $A$.  We labelled the inner component's two
gaussians $i=\{ 0, 1 \}$, and the single outer component as $i=2$.
Given the complexity of the line profile data
(Figure~\ref{fig:halphamodel} left column), the likelihood function
for this model is multimodal.  We therefore imposed the prior
constraints that $A_i \sim \mathcal{U}[0,1]$,
$\sigma_i/v_{\rm eq} \sim \mathcal{U}[0,1]$, and further assumed
$\mu_i(t) \sim
\mathcal{U}[
  K_{\rm inner}\sin(\phi(t)) - v_{\rm eq},
  K_{\rm inner}\sin(\phi(t)) + v_{\rm eq}
]$
for the inner two components, and
$\mu_2(t) \sim
\mathcal{U}[
  K_{\rm outer}\sin(\phi(t) + \pi) - 2v_{\rm eq},
  K_{\rm outer}\sin(\phi(t) + \pi) + 2v_{\rm eq}
]$
for the outer component.  This prior on the means mitigates
multimodality in the likelihood by requiring the mean velocity of each
component to be within a one or two $v_{\rm eq}$ of the time-variable
sinusoid suggested by visual inspection.  We fitted each component to
the data {\it independently} using scipy's non-linear least squares
\texttt{curve\_fit} implementation \citep{Virtanen2020}, and scaled the
resulting parameter covariance matrix by a constant factor to match
the sample variance of the residuals.  The resulting means,
amplitudes, and standard deviations for each component are given in
Table~\ref{tab:halphamodelparams}.

Caution is required in interpreting this model's results.  At some
epochs there are no significant spectral features around any component's
prior.  During such epochs, e.g.~the
``outer'' clump at $\phi$=1.07, the model fits noise, not signal.  At
other times, the model underfits.  For instance, the sudden blue rise
near $\phi$=0.2 is poorly described by a gaussian; the assumed
functional form is one of convenience.  Finally, the model fits the
$i=\{ 0, 1 \}$, and $i=2$ components independently.  At $\phi$=0.0 and
1.0, Figure~\ref{fig:halphamodel} suggests that the emission might
come from either the inner or outer components.  Physically however,
at this epoch the inner clump is in transit, and the outer clump is
passing behind the star.  At $\phi$=0.0, the double-peaked emission
profile also matches that seen shortly afterward (at $\phi$=0.14) when
the inner clump is viewed off-disk.  A physical interpretation of the
model would therefore discard the outer clump results at this
particular epoch, because its emission would be blocked by the star.

\begin{figure*}[!t]
  \centering
  \includegraphics[width=0.99\textwidth]{figures/sf7.pdf}
  \caption{{\bf a,} Orbits fit to mean radial velocities (RVs)
  extracted from H$\alpha$ profile fits in
  Figure~\ref{fig:halphamodel}.  Radial velocities on the vertical
  axis are in units of the equatorial velocity, $v_{\rm
  eq}$=130\,\kms.  Each marker denotes the best-fit gaussian mean at a
  given epoch.  Solid circles were adopted in the fits; transparent X
  markers were excluded due to reliability concerns (see text).  The
  inner double-peaked clump is shown in violet and orange; the outer
  clump is shown in green.  Five model draws from each posterior
  probability distribution are plotted.  {\bf b,} Idealized emissivity
  from mean model fits in Panel a, assuming a constant emission
  amplitude and velocity width $\sigma$=0.25\,$v_{\rm eq}$ and
  neglecting eclipses.}
  \label{fig:orbits}
\end{figure*}

Accounting for these caveats, we used the results from
Table~\ref{tab:halphamodelparams} to quantify the orbital periods and
velocity semi-amplitudes of the clumps.  Figure~\ref{fig:orbits}a
shows the results assuming circular orbits; considering the Bayesian
information criterion, we found no reason to prefer eccentric orbits.
By visually inspecting Figure~\ref{fig:halphamodel}, we
excluded epochs where our gaussian profile fitting failed to either
detect or else adequately represent the circumstellar emission.  We
then used NumPyro to sample the Gaussian likelihood for a Keplerian
orbit with the NUTS algorithm \citep{Phan2019}.  We used the
measurement uncertainties from each estimated mean radial velocity
value and included an additional jitter term in quadrature.  This
procedure yielded orbital periods and semi-amplitudes of
$P_0$=$3.92\pm0.03$\,hr, $K_0/v_{\rm eq}$=2.07$\pm$0.04;
$P_1$=$3.92\pm0.06$\,hr, $K_1/v_{\rm eq}$=2.88$\pm$0.10;
and $P_2$=$3.88\pm0.20$\,hr, $K_1/v_{\rm eq}$=3.88$\pm$0.25.
The periods for the inner double-peaked clump are therefore consistent
with the photometric $3.930\pm0.001$\,hr period within a precision of
two minutes.  The ``period'' for the outer clump is ambiguous because
the H$\alpha$ data only support the idea of a periodic orbit of
material well-fit by gaussian emission from $\phi$$\approx$0.5-1.0.
From $\phi$=0-0.2, there is no detectable emission, and from
$\phi$=0.2-0.4, the emission spans 1-4\,$v_{\rm eq}$ without taking a
clear gaussian shape.  While the outer-most edge of this
$\phi$=0.2-0.4 emission provides a plausible match to the expectation
of a circular orbit, the idea of invoking a particular functional form
for this component seems fine-tuned.  We instead emphasize that
although this emission is present, its variability in time is
inconsistent with the idea of a stable clump of material.  Additional
observations would be needed to conclusively determine whether or not
this component of the system is long-lived.

For Figure~\ref{fig:orbits}b, we then used the mean orbits from
Figure~\ref{fig:orbits}a to generate sinusoidal-in-time gaussians,
similar to the observations.  We assumed a constant emission amplitude
and velocity width $\sigma$=0.25\,$v_{\rm eq}$ for this exercise, and
normalized each gaussian to unit amplitude;  the colorbar in
Figure~\ref{fig:orbits}b thus masks the non-physical additive
contribution near the zero-crossing of velocity.  Compared to the
behavior of the data at $\phi$=0.2-0.5 (Figure~\ref{fig:spec}), this
is highly idealized.  Nonetheless, this exercise indicates that the
transit of the inner clump lasts $\approx$22\% of each cycle, with a
slight asymmetry around $\phi$=0.

{\it Physical dimensions of the emitting region}---The measured
velocity widths from the circumstellar emission contain information
about the size of the emitting region via the condition for rigid
corotation.  Consider a clump in cylindrical coordinates with
arbitrary radial extent $r$, azimuthal extent $\ell$, and height $z$.
A range of shapes, including an ``arc'' with $\ell \gg r$, a ``spoke''
with $r \gg \ell$, and a ``blob'' with $r \approx \ell \approx z$ are
all a priori possible.  However, at quadrature, the observed velocity
width of emission is sensitive to the radial extent of the
circumstellar material.  At mid-transit, the observed velocity width
is sensitive to the azimuthal extent.  An arc configuration would
minimize the observed velocity width $\sigma$ at quadrature, and
maximize it during transit, with $\gtrsim$100\,\kms\ variations in
between.  This is not observed.   The arc geometry, and by a similar
argument the spoke geometry, can thus be discarded.  

At quadrature, the inner clumps show $\sigma_i$$\approx$0.24\,$v_{\rm
eq}$, implying that 68\% of the emission comes from a volume with
length in the radial dimension $r_i$=$2\sigma_i /
\Omega$$=$0.48\,$R_\star$, and that 95\% of the emission comes from
within 0.96\,$R_\star$.  These values have relative uncertainties of
$\approx$5\%, based on the uncertainties in the measured velocity
dispersions.  The two inner clumps are centered at orbital distances
of 2.07\,$R_\star$ and 2.88\,$R_\star$.  There is therefore physical
overlap in their spatial distributions.  These two clumps could in
fact be a single clump with an optically thick H$\alpha$ line core.
Regardless of this nuance, the implication is that the full length in
the radial dimension of these two inner emitting clumps is
approximately equal to the star's diameter.  At mid-transit, these
inner clumps have a similar velocity dispersion, although with
greater uncertainty due to the differences between the $\phi$=0 and
$\phi$=1 transits (see Figure~\ref{fig:halphamodel}).
This suggests a 1$\sigma$ emission contour in the azimuthal dimension
with a length of $\ell$$\approx$0.5\,$R_\star$.

The vertical height of the emitting region is less constrained because
the system is consistent with being viewed edge-on.  However, one
constraint does come if one assumes that the TESS transit depth scales
with the projected H$\alpha$ emission area $\ell z$.  
More specifically, one can evaluate an ``effective area'' blocked by a
two-dimensional gaussian blob passing over a star by integrating the
local gaussian weight over the stellar surface.  For instance, if
$\ell$$\approx$$z$$\approx$0.24\,$R_\star$, then
$\iint \exp \left(
  -\left( x^2/2\sigma_x^2 + y^2/2\sigma_y^2 \right)
\right) {\rm d}x \, {\rm d}y$
suggests 11.5\% of the star being ``blocked'', a geometric factor
which would need to be in turn multiplied by an unknown opacity factor
to produce the observed transit depth ($\delta$$\approx$5\%).  In
general though, there is a degeneracy between $z$ and this opacity
factor; larger vertical heights are allowed for lower optical depths
in absorption, and vice-versa.  This constraint also implicitly
assumes that the optically thick material is well-mixed with the
hydrogen, which may not be accurate.


\section{Spectroscopic Variability}
\label{subsec:specvar}

\begin{figure*}[!t]
  \centering
  \includegraphics[width=\textwidth]{figures/sf5.pdf}
  \vspace{-0.25cm}
  \caption{{\bf Time evolution of the Keck/HIRES spectra
  from 17 February 2024.}  We made this plot by applying a windowed
  outlier rejection to remove cosmic rays and then smoothing each
  spectrum with a Gaussian filter.  The horizontal axis shows the
  velocity relative to each line's rest wavelength, normalized by the
  stellar equatorial velocity.  H$\alpha$ is the only Balmer line to
  show periodic variability similar to Figure~\ref{fig:spec}.
	The He 5876\,\AA\ line shows a time-dynamic blueshift.  Li 6708\,\AA\ shows
	no obvious absorption. }
  \label{fig:hirescuts}
\end{figure*}

Figure~\ref{fig:hirescuts} shows a few regions of
interest in the HIRES spectra, which cover 3650-7960\,\AA.  
H$\beta$ falls in an inter-order gap and was not observed.
Higher order Balmer lines including H$\gamma$ and H$\delta$
($n=5\rightarrow2$ and $n=6\rightarrow2$) do show variability outside
the line core.  However, this variability is not clearly periodic in the
same manner as the emission seen in H$\alpha$.  This could be because
there are insufficient hydrogen ions in the relevant excited states, or
because the spectra have lower precision in these bluer regions.  The Ca
HK doublet is also detected in emission, while the continuum near it is
not.  Chromospheric emission from the magnesium triplet is also
detected, but these lines are too blended to be usable.

Figure~\ref{fig:hirescuts} provides a novel view on the
blue excess that appeared at $\phi$$\approx$0.2 in H$\alpha$,
H$\gamma$, H$\delta$, and He 5876\,\AA.  At the same epoch, H$\alpha$
additionally shows a red clump of emission, and H$\gamma$ and
H$\delta$ are also broadened on the red wing.  The rise of this
emission in $<$15\,minutes might suggest a more sudden flow, rather
than a stable, periodic component.  For instance, a stellar flare
might be connected to such a sudden rise.  However, this idea seems
incompatible with the sinusoidal emission seen from
$\phi$$\approx$0.3-1.0, and with the fact that flares typically excite
iron lines in the blue HIRES orders, which are not observed.  More
time-series spectroscopy would be needed to clarify this type of
variability, in particular whether it is periodic or stochastic.
The gap in the TESS photometry (Figure~\ref{fig:fulllc}) additionally
clouds the interpretability of this event.


Finally, the Li\,\textsc{I} 6708\,\AA\ doublet, which shows no obvious
absorption, as well as the broad K\,\textsc{I} 7699\,\AA\ resonance
line are both visible in Figure~\ref{fig:hirescuts}.  In
the latter, narrow telluric absorption features overlap the blue wing
of the line.  Neither of these regions shows any notable variability.






\section{Properties of the Plasma and Magnetospheric Environment}
\label{subsec:gas}

The physical conditions inside the plasma clumps, in particular the
hydrogen number density, plasma temperature, ionization fraction, and
magnetic field strength can be estimated from the available data.  We
caution that the following estimates are order of magnitude
calculations that assume a simple uniform-density plasma: detailed
considerations of radiative transfer are a worthy topic for future
work, but are beyond the scope of this article.

Circumstellar H$\alpha$ emission might be sourced either from resonant
scattering of stellar H$\alpha$ photons, or from radiative
recombination.  We neglect scattering because Figure~\ref{fig:spec}
and Figure~\ref{fig:hirescuts} show the amplitude of the
circumstellar H$\alpha$ emission varying by a factor of $\approx$5, in
a manner uncorrelated with any variability in the chromospheric line
core.  The volume emissivity under case B recombination can be written
\begin{equation} j_{\rm H\alpha} = n_{\rm e} n_{\rm p} \alpha^{\rm
eff}_{\rm H\alpha} h \nu_{\rm H\alpha}, \end{equation} where $n_{\rm
e}$ and $n_{\rm p}$ are the electron and proton densities, and
$\alpha^{\rm eff}_{\rm H\alpha}$ is the effective recombination
coefficient, defined to include all recombination routes that produce
an H$\alpha$ photon.  For hydrogen with temperatures between
1,000-10,000\,K, $\alpha^{\rm eff}_{\rm H\alpha}$ is typically on the
order of $10^{-12}$ to $10^{-13}$\,cm$^3$\,s$^{-1}$
\citep{Hummer1987,Draine2011}.  Neglecting the effects of atoms other
than hydrogen, we can assume an ionization fraction $x$, such that
$n_{\rm e} = n_{\rm p} = x n_{\rm H}$, for $n_{\rm H}$ the hydrogen
number density.  Let $L_{\rm H\alpha} = j_{\rm H\alpha} V$, for $V$
the volume of the emitting hydrogen.  The luminosity of circumstellar
hydrogen emission, $L_{\rm H\alpha}$, is an observable: our SED
fitting routine yields $L_\star$$\approx$0.012\,$L_\odot$, which
implies that the stellar H$\alpha$ line radiates at
$\approx$1.0$\times$$10^{28}$erg\,s$^{-1}$.  The luminosity of the
clumps $L_{\rm H\alpha}$ are of order one tenth that of the star.  If
we approximate the emitting volume as a homogeneous sphere of radius
$r$, we can write
\begin{align}
  n_{\rm H} &= 
  1 \cdot 10^{11}\,{\rm cm}^{-3}
  \left(
    \frac{0.5}{x}
  \right)
  \left( 
    \frac{ L_{\rm H\alpha} }{ 10^{27}\,{\rm erg\,s^{-1}} }
    \frac{ 10^{-13}\,{\rm cm^{3}\,s^{-1}} }{ \alpha^{\rm eff}_{\rm H\alpha} }
  \right)^{1/2}
  \left(
    \frac{ 0.1\,R_\odot }{ r }
  \right)^{3/2}.
  \label{eq:numdensity}
\end{align}
For a uniform density clump, this suggests a total gas mass of $M_{\rm
gas}\approx 2\times10^{17}$\,g.  We emphasize that
Equation~\ref{eq:numdensity} in intended to provide only an order of
magnitude estimate for the number density implied by the observed
H$\alpha$ emission.  In detail, the effective recombination rate and
the ionization fraction each vary with density and temperature; a more
thorough estimate would iteratively solve the equations of detailed
balance and radiative transfer (e.g.~\citealt{CollierCameron1989}
Figure~8), and potentially also consider departures from local
thermodynamic equilibrium.

Finally, a constraint on the magnetic field strength at the site of
the clump follows from the requirement that the magnetic pressure
exceed the thermal pressure, $B_{\rm c}^2 / 8\pi > n_{\rm H} k T$.
Although we do not know the plasma temperature, if it were
significantly beyond 1,000-10,000\,K, we would either fully ionize the
hydrogen, or not ionize enough of it.  The field strength at the clump
must therefore exceed
\begin{equation}
  B_{\rm c} \gtrsim 1\,{\rm G}
  \left(
  \frac{n_{\rm H}}{1\times10^{11}\,{\rm cm}^{-3}}
  \frac{T}{3000\,{\rm K}}
  \right)^{1/2}.
\end{equation}
Given that the average surface magnetic field strengths of low-mass
stars have been measured to span hundreds to thousands of Gauss
\citep{Donati2009,Kochukhov2021,Reiners2022}, this bound is easily met
at orbital distances of 2-4\,$R_\star$.

\section{Upper and Lower Bounds on Dust}
\label{subsec:dust}

The material's composition -- either pure plasma, or a dusty plasma --
is not known.  The idea of dust being present seems plausible given
observations of chromatic transits in analogous objects
\citep{Tanimoto2020,Gunther2022,Koen2023}.  However, this scenario is
highly constrained.  An upper limit on the amount of hot dust follows
from the lack of an infrared excess.  A lower limit follows if one
assumes that most of the broadband optical depth comes from dust
absorption and scattering, rather than any radiative processes
associated with the plasma.

Regarding the upper limit, Figure~\ref{fig:sed} shows
the SED.  While AllWISE \citep{Cutri2014} yielded a confident W3
detection (9.8$\sigma$) consistent with the photospheric extrapolation
from bluer bandpasses, the W4 extraction yielded only a marginal
indication (1.7$\sigma$) of detectable flux.  Similar to other CPVs
\citep{Stauffer2017,Bouma2024}, the photometric uncertainties from WISE
W1 and W2 allow at most a $\lesssim$2\% excess at 3-5\,$\mu$m relative
to the stellar photosphere, and a $\lesssim$5\% excess at 10\,$\mu$m
(W3).  To estimate the implied mass bound, we assume a dust
temperature $T_{\rm d}$=1500\,K, typical for dust near the star (see
\citealt{Zhan2019} for discussion regarding dust sublimation).  We then
treat emission from the dust and star as Planck functions, and require
$L_{\rm d} < f L_\star$, where the factor $f$ is set by the
photometric precision of WISE and $L_{\rm d}$ is the bolometric dust
luminosity.  Given the reported uncertainties, we numerically find
$f<6\cdot$10$^{-3}$.  From the Stefan-Boltzmann law we can then write
$A_{\rm d} < f (T_\star / T_{\rm d})^4 Q_{\rm em}^{-1} (4\pi
R_\star)^2$, for $A_{\rm d}$ the total emitting surface area of the
dust, and $Q_{\rm em}$ an emission efficiency parameter.  Converting
this constraint to a dust mass requires an assumption regarding the
grain properties.  We assume a grain density $\rho_{\rm
d}$=3\,g\,cm$^{-3}$ typical for silicate grains, a fixed grain size
$a$=1\,$\mu$m, and no self-absorption.  This enables the assumption
that $A_{\rm d} = N \pi a^2$, for $N$ the total number of dust grains.
This in turn yields an upper limit on the dust mass of
\begin{equation}
  M_{\rm dust} \lesssim 4 \cdot 10^{17}\, {\rm g}\ 
  \left( \frac{f}{6\cdot10^{-3}} \right)
  \left( \frac{T_\star}{3000\,{\rm K}} \frac{1500\,{\rm K}}{T_{\rm d}} \right)^4
  \left( \frac{Q_{\rm em}}{1} \right)^{-1}
  \left( \frac{R_\star}{0.4\,R_\odot} \right)^2
  \left( \frac{a}{1\,\mathrm{\mu m}} \right)
  \left( \frac{\rho_{\rm d}}{3\,\mathrm{g\,cm^{-3}}} \right).
\end{equation}

The analogous lower limit follows from requiring the optical depth
from absorption and scattering $\tau$ to be at least unity.  The
optical depth can be written $\tau = n \sigma \ell$, where $\sigma$ is
the cross-section, $n$ is the number density, $\ell$ is the path
length.  For spherical dust grains in the optical, $\sigma = Q_{\rm
ext} \pi a^2$, where $Q_{\rm ext}$ is the extinction efficiency
parameter, tabulated e.g.~by \citet{Croll2014} in their Figure 13.
\citet{Sanderson2023} calculated the relevant cloud mass for this
problem assuming a spherical dust clump of size $r$, and they found
\begin{equation}
  M_{\rm dust} \gtrsim 2 \cdot 10^{15}\, {\rm g}\ 
  \left( \frac{\tau}{1} \right)
  \left( \frac{Q_{\rm ext}}{3} \right)^{-1}
  \left( \frac{r}{0.1\,R_\star} \frac{R_\star}{0.4\,R_\odot} \right)^2
  \left( \frac{a}{1\,\mathrm{\mu m}} \right)
  \left( \frac{\rho_{\rm d}}{3\,\mathrm{g\,cm^{-3}}} \right).
\end{equation}

Three relevant objects for comparison include solar prominences,
planetesimals, and comets.  Prominences of the Sun have gas masses of
$10^{14}$\,g-$10^{16}$\,g \citep{VialEngvold2015}.  A planetesimal of
mass $\approx$10$^{15}$\,g with a bulk density of 1\,g\,cm$^{-3}$
would have a diameter of order one kilometer.  Halley's comet has a
mass of order $10^{17}$\,g \citep{Rickman1989}, of which
$\sim$$10^{14}$\,g is shed per orbit, most of which inspirals toward
the Sun due to Poynting-Robertson drag.

To summarize, if dust is responsible for the broadband variability of
CPVs, it would need to be concentrated in clumps with masses in the
range of $10^{15}$-$10^{17}$\,g.  Given $M_{\rm
gas}$$\approx$2$\times$10$^{17}$\,g from Appendix~\ref{subsec:gas}, the
allowed dust masses imply $M_{\rm gas}/M_{\rm dust}$ ranges of 1-100.
More careful measurements of this ratio---in particular by inferring
the dust mass through high precision infrared
spectrophotometry---could provide a path for distinguishing the
scenario of a trapped stellar outflow from an accumulation of
externally-sourced material.  While there are several plausible
external sources, feeding through a low-mass disk in particular cannot
be ruled out based on typical disk depletion times \citep{Haisch2001}.
Observations of infrared excesses and accretion signatures in low-mass
stars tens of millions of years old suggest a broad lifetime
distribution for such disks
\citep{Silverberg2020,Lee2020,Gaidos2022,Pfalzner2024}.


%%%%%%%%%%%%%%%%%%%%%%%%
% Supplementary Tables %
%%%%%%%%%%%%%%%%%%%%%%%%

\begin{table}
    \centering
    \begin{tabular}{lcr}
    \hline 
    \hline
    Parameter & Host & Source \\
    \hline 
    \multicolumn{3}{c}{Identifiers} \\
    \hline
    TIC & 141146667 & TESS \\
    Gaia & 860453786736413568 & Gaia\ DR3 \\
    \hline
    \multicolumn{3}{c}{Astrometry \& Radial Velocity} \\ 
    \hline
    $\alpha_{2000}$ & 11:05:15.09   & Gaia\ DR3 \\
    $\delta_{2000}$ & +59 15 05.57  & Gaia\ DR3 \\
    $\mu_{\alpha}$ (mas yr$^{-1}$ ) & -73.933 $\pm$ 0.022 & Gaia\ DR3 \\
    $\mu_{\delta}$ (mas yr$^{-1}$ ) &  32.262 $\pm$ 0.024 & Gaia\ DR3 \\
    $\pi$ (mas)                     &  17.324 $\pm$ 0.025 & Gaia\ DR3 \\
    RUWE                            &  1.23               & Gaia\ DR3 \\
    RV (\kms)                       & 0.61 $\pm$ 1.47     & HIRES \\
    \hline
    \multicolumn{3}{c}{Photometry} \\
    \hline
    $TESS$ (mag)                    & 13.283 $\pm$ 0.010 & TIC8\     \\
    $G$ (mag)                       & 14.701 $\pm$ 0.002 & Gaia\ DR3 \\
    $G_{\rm BP}$ (mag)              & 16.664 $\pm$ 0.008 & Gaia\ DR3 \\
    $G_{\rm RP}$ (mag)              & 13.398 $\pm$ 0.006 & Gaia\ DR3 \\
    $G_{\rm BP}$-$G_{\rm RP}$ (mag) &  3.276 $\pm$ 0.010 & Gaia\ DR3 \\
    $J$ (mag)                       & 11.401 $\pm$ 0.022 & 2MASS     \\
    $H$ (mag)                       & 10.793 $\pm$ 0.021 & 2MASS     \\
    $K_s$ (mag)                     & 10.473 $\pm$ 0.016 & 2MASS     \\
    $W1$ (mag)                      & 10.276 $\pm$ 0.023 & ALLWISE   \\ % Cutri+2013 II/328/allwise
    $W2$ (mag)                      & 10.070 $\pm$ 0.020 & ALLWISE   \\
    $W3$ (mag)                      &  9.838 $\pm$ 0.045 & ALLWISE   \\
    %$W4$ (mag)                      &  8.392 $\pm$ NOUNC & ALLWISE \\
	  %
    % NOTE: ROSAT is an emitter.  GALEX detection.
    % ROSAT: 3.25e-13 mW/m^2 = erg/cm^2/s
    % Ofek+2011 1.4Ghz emitter.  Helfand+2015 ditto.  Bruzewski+2021.
    \hline
    \multicolumn{3}{c}{Physical Properties} \\
    \hline
    $T_{\rm eff}$ (K) & 2972 $\pm$ 40 & SED fit \\
    $R_\star$ ($R_{\odot}$) & 0.42 $\pm$ 0.02 & SED fit  \\
    $L_\star$ ($L_\odot$) & 0.0126 $\pm$ 0.0012 & SED fit \\
    $P_{\rm rot}$ (hours) & 3.930 $\pm$ 0.001 & TESS \\ 
    $v_{\rm eq}$ (\kms)  &  130$\pm$4  & Derived \\
    $v_{\rm eq} \sin i_\star$ (\kms) & 138 $\pm$ 8 & HIRES \\
    $i_\star$($^\circ$) & $>$63 & Derived \\
    $A_V$ (mag) & 0 & \citep{Green2019} \\
    $M_\star$ ($M_{\odot}$)  & 0.22 $\pm$ 0.02  & PARSEC \citep{Chen2014}\\
    ${\rm EW}_{\rm Li}$ (m\AA) & $<$107 & HIRES (2$\sigma$)\\
    $t_{\rm CAMD}$ (Myr) & 30-150 &  Gaia DR3 \\
    $t_{\rm Li,emp}$ (Myr) & $>$20 &  HIRES, \citep{Jeffries2023} \\
    $t_{\rm Li,th}$ (Myr) & $>$35 &  HIRES, \citep{Feiden2016} \\
    $t_{\rm adopted}$ (Myr) & 35-150 &  -- \\
    \hline
    \end{tabular}
		\caption{Properties of TIC\,141146667.  References:
    Gaia DR3 \citep{GaiaDR3}, TESS \citep{Ricker2015},
    TIC8 \citep{Stassun2019}, 2MASS \citep{Skrutskie2006}, ALLWISE
    \citep{Cutri2014}.}
    \label{tab:stparams}
\end{table}


\begin{table}
  \centering
  \begin{tabular}{lcr}
  \hline 
  \hline 
  Time [BJD$_{\rm TDB}$] & RV (\kms) & $\sigma_{\rm RV}$ (\kms) \\
  % from tables/tab_TIC141146667_rel_rv.tex and
  % tables/tab_TIC141146667_times_rounded.csv
  \hline 
  2460357.954919 & 2.73 & 5.86 \\
  2460357.965845 & -4.40 & 2.37 \\
  2460357.976770 & -0.19 & 2.64 \\
  2460357.987698 & 3.84 & 2.87 \\
  2460357.998619 & 7.53 & 7.53 \\
  2460358.009538 & -1.98 & 1.44 \\
  2460358.020462 & 1.02 & 1.21 \\
  2460358.031383 & 0.64 & 7.03 \\
  2460358.042306 & -2.91 & 2.71 \\
  2460358.053228 & 8.93 & 6.75 \\
  2460358.064154 & 5.95 & 8.84 \\
  2460358.075075 & -2.25 & 3.06 \\
  2460358.085996 & 1.84 & 1.34 \\
  2460358.096918 & 2.41 & 8.24 \\
  2460358.107839 & -7.04 & 3.94 \\
  2460358.118760 & -2.24 & 3.07 \\
  2460358.129683 & -2.83 & 7.55 \\
  2460358.140606 & -0.59 & 2.26 \\
  2460358.151527 & 1.84 & 2.91 \\
  2460358.162448 & 4.54 & 3.95 \\
  2460358.173368 & 6.21 & 12.14 \\
  \hline
  \end{tabular}
  \caption{TIC\,141146667 radial velocities relative to the systemic
  velocity based on the 7699\,\AA\ resonance line and TiO bandheads.}
  \label{tab:rv}
\end{table}


\begin{sidewaystable}
  \footnotesize
  \centering
  \begin{tabular}{lccccccccc}
  % from results/halpha_to_rv_timerseries/multigauss_parametertable.tex
  \hline
  BTJD & $\mu_0$ & $\mu_1$ & $\mu_2$ & $\sigma_0$ & $\sigma_1$ & $\sigma_2$ & $A_0$ & $A_1$ & $A_2$ \\
  \hline
  3357.960099 & -0.725 $\pm$ 0.028 & -1.416 $\pm$ 0.026 & 0.177 $\pm$ 0.019 & 0.39 $\pm$ 0.03 & 0.18 $\pm$ 0.03 & 0.59 $\pm$ 0.02 & 0.51 $\pm$ 0.01 & 0.23 $\pm$ 0.04 & 0.29 $\pm$ 0.01 \\
  3357.970746 & -0.192 $\pm$ 0.007 & 0.498 $\pm$ 0.010 & 0.100 $\pm$ 0.014 & 0.23 $\pm$ 0.01 & 0.24 $\pm$ 0.01 & 0.55 $\pm$ 0.01 & 0.50 $\pm$ 0.01 & 0.37 $\pm$ 0.01 & 0.42 $\pm$ 0.01 \\
  3357.981672 & 0.895 $\pm$ 0.051 & 1.771 $\pm$ 0.076 & -1.470 $\pm$ 0.113 & 0.40 $\pm$ 0.02 & 0.40 $\pm$ 0.10 & 0.09 $\pm$ 0.11 & 0.67 $\pm$ 0.05 & 0.34 $\pm$ 0.04 & 0.03 $\pm$ 0.03 \\
  3357.993524 & 1.536 $\pm$ 0.011 & 2.202 $\pm$ 0.015 & -3.787 $\pm$ 0.049 & 0.25 $\pm$ 0.01 & 0.29 $\pm$ 0.01 & 0.72 $\pm$ 0.05 & 0.55 $\pm$ 0.01 & 0.46 $\pm$ 0.01 & 0.05 $\pm$ 0.00 \\
  3358.003594 & 1.974 $\pm$ 0.007 & 2.588 $\pm$ 0.006 & -2.749 $\pm$ 0.020 & 0.26 $\pm$ 0.01 & 0.22 $\pm$ 0.00 & 1.00 $\pm$ 0.02 & 0.64 $\pm$ 0.01 & 0.67 $\pm$ 0.01 & 0.43 $\pm$ 0.01 \\
  3358.015249 & 2.027 $\pm$ 0.003 & 2.666 $\pm$ 0.003 & -2.370 $\pm$ 0.015 & 0.24 $\pm$ 0.00 & 0.20 $\pm$ 0.00 & 1.00 $\pm$ 0.01 & 0.62 $\pm$ 0.00 & 0.60 $\pm$ 0.00 & 0.56 $\pm$ 0.00 \\
  3358.024971 & 1.790 $\pm$ 0.004 & 2.469 $\pm$ 0.006 & -2.138 $\pm$ 0.011 & 0.24 $\pm$ 0.00 & 0.26 $\pm$ 0.00 & 0.77 $\pm$ 0.01 & 0.53 $\pm$ 0.00 & 0.38 $\pm$ 0.00 & 0.31 $\pm$ 0.00 \\
  3358.036522 & 0.584 $\pm$ 0.008 & 1.399 $\pm$ 0.009 & -1.584 $\pm$ 0.009 & 0.18 $\pm$ 0.01 & 0.40 $\pm$ 0.01 & 0.89 $\pm$ 0.01 & 0.35 $\pm$ 0.01 & 0.39 $\pm$ 0.00 & 0.62 $\pm$ 0.00 \\
  3358.047668 & 0.328 $\pm$ 0.014 & 0.685 $\pm$ 0.010 & -0.969 $\pm$ 0.017 & 0.07 $\pm$ 0.01 & 0.13 $\pm$ 0.01 & 0.46 $\pm$ 0.02 & 0.14 $\pm$ 0.02 & 0.28 $\pm$ 0.02 & 0.52 $\pm$ 0.02 \\
  3358.058824 & -0.646 $\pm$ 0.005 & -1.474 $\pm$ 0.066 & 0.891 $\pm$ 0.079 & 0.23 $\pm$ 0.00 & 0.25 $\pm$ 0.07 & 0.14 $\pm$ 0.08 & 0.51 $\pm$ 0.01 & 0.10 $\pm$ 0.01 & 0.09 $\pm$ 0.04 \\
  3358.069519 & -0.388 $\pm$ 0.104 & -1.669 $\pm$ 0.023 & 2.646 $\pm$ 0.020 & 0.40 $\pm$ 0.07 & 0.40 $\pm$ 0.02 & 0.53 $\pm$ 0.02 & 0.47 $\pm$ 0.04 & 0.33 $\pm$ 0.01 & 0.22 $\pm$ 0.01 \\
  3358.080468 & -1.748 $\pm$ 0.009 & -2.483 $\pm$ 0.010 & 3.526 $\pm$ 0.009 & 0.25 $\pm$ 0.01 & 0.33 $\pm$ 0.01 & 0.55 $\pm$ 0.01 & 0.28 $\pm$ 0.01 & 0.32 $\pm$ 0.00 & 0.32 $\pm$ 0.00 \\
  3358.091255 & -2.119 $\pm$ 0.007 & -2.868 $\pm$ 0.007 & 3.960 $\pm$ 0.005 & 0.26 $\pm$ 0.01 & 0.23 $\pm$ 0.01 & 0.40 $\pm$ 0.01 & 0.34 $\pm$ 0.01 & 0.29 $\pm$ 0.01 & 0.39 $\pm$ 0.00 \\
  3358.101671 & -2.019 $\pm$ 0.006 & -2.731 $\pm$ 0.008 & 3.758 $\pm$ 0.005 & 0.20 $\pm$ 0.01 & 0.24 $\pm$ 0.01 & 0.43 $\pm$ 0.01 & 0.29 $\pm$ 0.01 & 0.25 $\pm$ 0.01 & 0.23 $\pm$ 0.00 \\
  3358.112713 & -1.581 $\pm$ 0.020 & -2.141 $\pm$ 0.043 & 2.867 $\pm$ 0.019 & 0.19 $\pm$ 0.02 & 0.29 $\pm$ 0.03 & 0.40 $\pm$ 0.02 & 0.18 $\pm$ 0.02 & 0.13 $\pm$ 0.01 & 0.12 $\pm$ 0.00 \\
  3358.123928 & -0.624 $\pm$ 0.006 & -1.312 $\pm$ 0.076 & 0.824 $\pm$ 0.012 & 0.09 $\pm$ 0.01 & 0.22 $\pm$ 0.08 & 0.44 $\pm$ 0.01 & 0.18 $\pm$ 0.01 & 0.02 $\pm$ 0.01 & 0.32 $\pm$ 0.01 \\
  3358.135236 & -0.012 $\pm$ 0.011 & -0.208 $\pm$ 0.012 & -0.192 $\pm$ 0.012 & 0.03 $\pm$ 0.01 & 0.11 $\pm$ 0.01 & 0.12 $\pm$ 0.01 & 0.12 $\pm$ 0.04 & 0.25 $\pm$ 0.02 & 0.25 $\pm$ 0.02 \\
  3358.146555 & 1.007 $\pm$ 0.100 & 1.622 $\pm$ 0.048 & -1.789 $\pm$ 0.045 & 0.06 $\pm$ 0.11 & 0.31 $\pm$ 0.06 & 0.04 $\pm$ 0.05 & 0.02 $\pm$ 0.04 & 0.12 $\pm$ 0.02 & 0.05 $\pm$ 0.05 \\
  3358.156693 & 1.603 $\pm$ 0.018 & 2.444 $\pm$ 0.024 & -4.408 $\pm$ 0.197 & 0.14 $\pm$ 0.02 & 0.23 $\pm$ 0.03 & 1.00 $\pm$ 0.17 & 0.14 $\pm$ 0.02 & 0.14 $\pm$ 0.01 & 0.05 $\pm$ 0.00 \\
  3358.169147 & 1.543 $\pm$ 0.007 & 2.152 $\pm$ 0.013 & -3.885 $\pm$ 0.031 & 0.14 $\pm$ 0.01 & 0.39 $\pm$ 0.01 & 1.00 $\pm$ 0.04 & 0.23 $\pm$ 0.01 & 0.27 $\pm$ 0.00 & 0.10 $\pm$ 0.00 \\
  3358.177978 & 1.588 $\pm$ 0.037 & 2.828 $\pm$ 0.005 & -4.378 $\pm$ 0.053 & 0.40 $\pm$ 0.03 & 0.02 $\pm$ 0.01 & 0.75 $\pm$ 0.06 & 0.33 $\pm$ 0.01 & 0.11 $\pm$ 0.03 & 0.12 $\pm$ 0.01 \\
  \hline
  \end{tabular}
  \caption{Best-fit parameters from multi-gaussian model fit to H$\alpha$ line
  profiles.  Units for $\mu_i$ and $\sigma_i$ are $v_{\rm eq}$
  (130\,\kms). These results should be treated with caution; only a
  subset of the spectral epochs yielded statistically significant
  detections of the circumstellar emission (see
  Figure~\ref{fig:halphamodel}).}
  \label{tab:halphamodelparams}
\end{sidewaystable}


%%%%%%%%%%%%%%%%%%%%%%%%%%%%%%%%%%%%%%%%%%
%%%%%%%%%%%%%%%%%%%%%%%%%%%%%%%%%%%%%%%%%%

\clearpage

\bibliographystyle{aasjournal}
\bibliography{cpvbib.bib}

%%%%%%%%%%%%%%%%%%%%%%%%%%%%%%%%%%%%%%%%%%
%%%%%%%%%%%%%%%%%%%%%%%%%%%%%%%%%%%%%%%%%%

\end{document}


